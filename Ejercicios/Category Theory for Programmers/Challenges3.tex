\documentclass[11pt]{article}
%Gummi|065|=)
\usepackage[utf8x]{inputenc}
\usepackage[margin=1in]{geometry} 
\usepackage{amsmath,amsthm,amssymb}
\usepackage[spanish]{babel}
\usepackage{mathpazo}
\usepackage{tikz}
\usepackage{tikz-cd}
\usetikzlibrary{babel}
\SetUnicodeOption{mathletters}
\SetUnicodeOption{autogenerated}
\setlength\parindent{0pt}

\usepackage{listings}
\usepackage{xcolor}
\lstset { %
    language=C++,
    backgroundcolor=\color{black!5},
    basicstyle=\footnotesize,
}

\begin{document}
\begin{enumerate}
\item \textit{Generate a free category from:}
\begin{enumerate}
\item \textit{A graph with one node and no edges}

Obtenemos la categoría monoidal con sólo una flecha identidad.

\item \textit{A graph with one node and one (directed) edge}

Obtenemos la categoría monoidal con una flecha por cada composición de la flecha original con sí misma.

\item \textit{A graph with two nodes and a single arrow between them}

Obtenemos la categoría de dos objetos, dos flechas identidad y una flecha entre los objetos.

\item \textit{A graph with a single node and 26 arrows marked with the letters of the alphabet: a,b,c,\dots,z}.

Obtenemos la categoría monoidal con una flecha por cada palabra en el alfabeto a,b,\dots,z.
\end{enumerate}

\item \textit{What kind of order is this?}
\begin{enumerate}
\item \textit{A set of sets with the inclusion relation $A$ is included in $B$ if every element of $A$ is also an element of $B$.}

Orden parcial.

\item \textit{C++ types with the following subtyping relation: \texttt{T1} is a subtype of \texttt{T2} if a pointer to \texttt{T1} can be passed to a function that expects a pointer to \texttt{T2} without triggering a compilation error.}

Según la documentación: pointers in general allow the following conversions:
\begin{itemize}
\item Null pointers can be converted to pointers of any type
\item Pointers to any type can be converted to void pointers.
\item Pointer upcast: pointers to a derived class can be converted to a pointer of an accessible and unambiguous base class, without modifying its const or volatile qualification.
\end{itemize}
Con estas normas deducimos que es un orden parcial    
\end{enumerate}

\item \textit{Considering that \texttt{Bool} is a set of two values \texttt{True} and \texttt{Flase}, show that it forms two (set-theoretical) monoids with respect to, respectiely, operator $\&\&$ (AND) and $||$ (OR).}

Basta ver por casos que $\&\&$ y $||$ respeta la asociatividad.

\item \textit{Represent the \texttt{Bool} monoid with the AND operator as a category: List the morphisms and their rules of composition.}

\begin{tikzcd}
\texttt{Bool} \arrow[loop above, "T\&"] \arrow[loop below, "F\&"]
\end{tikzcd}

Tenemos los morfismos $T\&$ y $F\&$. Véase que $T\&$ es la identidad y que $F\&$ es idempotente.

\item \textit{Represent addition modulo 3 as a monoid category.}

\begin{tikzcd}
\mathbb{Z}/3\mathbb{Z} \arrow[loop above, "0+"] \arrow[loop right, "1+"] \arrow[loop below, "2+"]
\end{tikzcd}

Tenemos que $(0+)$ es la identidad. $(1+) \circ (1+) = (2+)$, $(2+) \circ (1+) = (1+) \circ (2+) = (0+)$ y $(2+) \circ (2+) = (1+)$.
\end{enumerate}
\end{document}

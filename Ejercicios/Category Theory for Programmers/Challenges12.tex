\documentclass[11pt]{article}
%Gummi|065|=)
\usepackage[utf8x]{inputenc}
\usepackage[margin=1in]{geometry} 
\usepackage{amsmath,amsthm,amssymb}
\usepackage[spanish]{babel}
\usepackage{mathpazo}
\usepackage{verbatim}
\usepackage{tikz}
\usepackage{tikz-cd}
\usetikzlibrary{babel}
\SetUnicodeOption{mathletters}
\SetUnicodeOption{autogenerated}
\setlength\parindent{0pt}

\usepackage{listings}
\usepackage{xcolor}
\lstset { %
    language=C++,
    backgroundcolor=\color{black!5},
    basicstyle=\footnotesize,
}

\usepackage{amsmath}
\begin{document}
\begin{enumerate}
\item \textit{How would you describe a pushout in the category of C++ classes?}

Un co-cono sería una superclase $D$ de otras clases $A$ y $B$ que son a su vez superclase de una clase $C$. Para que $D$ sea pullback, cualquier otra superclase $D'$ de $A$ y $B$ debe ser superclase (o al menos \textit{casteable}) de $D$. De aquí deducimos que $D$ debe ser una clase virtual pura con declaraciones como métodos virtuales de todos los métodos presentes en $A$ y $B$ simultáneamente.

\item \textit{Show that the limit of the indentity functor $Id :: C \to C$ es the initial object.}

Para que un objeto $c \in C$ sea cono bajo este functor, es necesario que haya una transformación natural de $Δ_c$ a $Id$. Esto implica que haya morfismos de $c$ a cualquier objeto de $C$. Es decir, la categoría de conos estará formada por todos los objetos que tienen morfismos a los demás objetos de $C$. Si $C$ tiene un objeto inicial, esta categoría estará formada por sólo el objeto inicial. Luego el límite, el objeto terminal de esta categoría, es tivialmente el único objeto, el objeto inicial de $C$.

\item \textit{Subsets of a given set form a category. A morphism in that category is defined to be an arrow connecting two sets if the first is the subset of the second. What is a pullback of two sets in such a category? What's a pushout? What are the initial and terminal objects?}

Consideramos el cospan $a \rightarrow b \leftarrow c$ en una categoría de subconjuntos de $A$, es decir, $a$, $b$ y $c$ son subconjuntos tales que $a \subseteq b \supseteq c$. Un cono será cualquier subconjunto de $a \cap c$. El límite será el propio conjunto $a \cap c$. Luego el pullback de dos subconjuntos es su intersección. Similarmente, el pushout es su unión. El objeto inicial es el conjunto vacío, mientras el objeto terminal es el propio $A$.

\item \textit{Can you guess what a coequalizer is?}

Dado dos morfismos $f$ y $g$ de $a$ a $b$ en una categoría $\mathcal{C}$, su coequalizer será un objeto $c$ con un morfismo $q : b \to c$ tal que:
\[ q \circ f = q \circ g \]
con la propiedad universal. Es decir, $q$ iguala todos los elementos que son distintos por la imagen de $f$ y $g$. Cualquier otro objeto $c'$ con morfismo $q' : b \to c$ tal que $q' \circ f = q' \circ g$ debe cumplir que existe un único morfismo $h : c \to c'$ tal que $q' = h \circ q$.

\item \textit{Show that, in a category with a terminal object, a pullback towards the terminal object is a product.}

Consideramos el cospan $a \xrightarrow{f} t \xleftarrow{g} b$ con $a$, $b$ y $t$ en la categoría $C$ con $t$ objeto terminal.

\item \textit{Similarly, show that a pushout from an initial object (if one exists) is the coproduct.}
\end{enumerate}
\end{document}

\documentclass[11pt]{article}
%Gummi|065|=)
\usepackage[utf8x]{inputenc}
\usepackage[margin=1in]{geometry} 
\usepackage{amsmath,amsthm,amssymb}
\usepackage[spanish]{babel}
\usepackage{mathpazo}
\usepackage{verbatim}
\usepackage{tikz}
\usepackage{tikz-cd}
\usetikzlibrary{babel}
\SetUnicodeOption{mathletters}
\SetUnicodeOption{autogenerated}
\setlength\parindent{0pt}

\usepackage{listings}
\usepackage{xcolor}
\lstset { %
    language=C++,
    backgroundcolor=\color{black!5},
    basicstyle=\footnotesize,
}

\usepackage{amsmath}
\begin{document}
\begin{enumerate}
\item \textit{Show that the hom-functors map identity morphism in $C$ to corresponding identity functions in $Set$.}

Sea $a$ un objeto en $C$. Tenemos que $C(a,id)$ es un morfismo desde $C(a,a)$ a $C(a,a)$ dado por $h \mapsto id \circ h = h$, es decir, $C(a,id)=id_{C(a,a)}$.

\item \textit{Show that \texttt{Maybe} is not representable.}

Seguimos el método usado en el texto para listas. Supongamos que \texttt{Maybe} fuera representable, entonces existe para en particular \texttt{a :: Int} tenemos que debe existir una transformación natural
\begin{verbatim}
alpha :: forall x. (Int -> x) -> Maybe x
alpha h = fmap h (Just 1)
\end{verbatim}
pero al buscar inversa:
\begin{verbatim}
beta :: forall x. Maybe x -> (Int -> x)
\end{verbatim}
nos encontramos que \texttt{beta Nothing :: Int -> x}, que es absurdo.

\item \textit{Is the \texttt{Reader} functor representable?}

Sí, pues \texttt{Reader} es precisamente el hom-functor, que es, por definición, representable.

\item \textit{Using \texttt{Stream} representation, memoize a function that squares its argument.}

\begin{verbatim}
square :: Integer -> Integer
square = index (tabulate (\n -> n^2) :: Stream Integer)
\end{verbatim}

\item \textit{Show that \texttt{tabulate} and \texttt{index} for \texttt{Stream} are indeed the inverse of each other. (Hint: use induction).}

Lo demostramos por inducción, el caso $n=0$:

\begin{verbatim}
index (tabulate f) 0
= { definition of tabulate }
  index (Const (f 0) (tabulate (f . (+1)))) 0
= { definition of index }
  f 0
\end{verbatim}

Supongamos que es cierta la igualdad para $n$, entonces:

\begin{verbatim}
index (tabulate f) (n+1)
= { definition of tabulate }
  index (Const (f 0) (tabulate (f . (+1)))) (n+1)
= { definition of index }
  if n+1 == 0 then f 0 else index (Const (f 0) (tabulate (f . (+1)))) n
= { n+1 > 0 }
  index (Const (f 0) (tabulate (f . (+1)))) n
= { induction hypothesis }
  (f . (+1)) n
= { definition of (.) }
  f (n+1)
\end{verbatim}

Luego \texttt{index . tabulate = id}. Por otro lado:

\begin{verbatim}
tabulate (index (Cons b bs))
= { definition of tabulate }
  Cons (index (Cons b bs) 0) (tabulate (index (Cons b bs) . (+1)))
= { definition of index }
  Cons b (tabulate (index (Cons b bs) . (+1)))
= { η introduction }
  Cons b (tabulate (\n -> index (Cons b bs) (n+1)))
= { definition of index with n+1 > 0 }
  Cons b (tabulate (\n -> index bs n))
= { η elimination }
  Cons b (tabulate (index bs))
\end{verbatim}

Por inducción, se deduce que \texttt{tabulate . index = id}.

\item \textit{The functor \texttt{Pair a = Pair a a} is representable. Can you guess the type that represents it? Implement \texttt{tabulate} and \texttt{index}}.

\begin{verbatim}
data Pair a = Pair a a
instance Representable Pair where
    type Rep Pair = Bool
    tabulate f = Pair (f False) (f True)
    index (Pair x _) False = x
    index (Pair _ y) True = y
\end{verbatim}
\end{enumerate}
\end{document}

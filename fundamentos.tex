\chapter{Fundamentos}
\section{Categorías}

% Abramsky - Logic and categories as tools for building theories
La idea empieza por el concepto de categoría, que resulta sorprendentemente ubicua en la matemática.
En principio, una categoría es sólo una colección de objetos y flechas entre dichos objetos donde se define una composición de flechas.
Veremos que algunas de la categorías más interesantes son las relacionadas con las estructuras matemáticas, como $\Top$, la categoría de espacios topológicos y funciones continuas, o $\Grp$, la categoría de grupos y homomorfismos.
En estos casos, la teoría de categorías sirve como un marco de las teorías matemáticas.
Las aplicaciones entre categorías --functores-- ilustran las conexiones entre distintas teorías matemáticas.
Aunque los axiomas que pediremos a las categorías no son muchos, resulta que podemos crear una gran cantidad de resultados interesantes y muy genéricos.

\begin{definition}
Una \newterm{categoría} $\cat$ es:
\begin{itemize}
\item Una colección de \index{objeto}\emph{objetos} $O(\cat)$.
\item A cada par de objetos $A$, $B$ en $O(\cat)$, una colección de \index{morfismo}\emph{morfismos} $\cat(A,B)$ de $A$ a $B$.
Un morfismo $f$ en $\cat(A,B)$ se denotará $f \colon A \to B$.
\item A cada par de morfismos $f \colon A \to B$, $g \colon B \to C$, un morfismo de $g \circ f \colon A \to C$ llamado \newterm{morfismo composición} de $f$ y $g$.
\end{itemize}
de manera que:
\begin{itemize}
\item Para cada objeto $A \in O(\cat)$, existe un morfismo $\id_A \in \cat(A,A)$ que llamamos \newterm{morfismo identidad}.
\item Para todo morfismo $f \colon A \to B$:
\[ \id_B \circ f = f \circ \id_A = f \]
\item Asociatividad de la composición: Para toda tripleta de morfismos $f \colon A \to B$, $g \colon B \to C$, $h \colon C \to D$:
\begin{equation}\label{cat:3} (h \circ g) \circ f = h \circ (g \circ f) \end{equation}
\end{itemize}
\end{definition}

Una notación habitual es usar $Hom(A,B)$ para referirse a la colección de morfismos entre $A$ y $B$, lo que hemos llamado $\cat(A,B)$.

A menudo usaremos diagramas para expresar visualmente ciertas propiedades.
En Teoría de Categoría, existe una definición precisa de diagrama, pero por ahora, no tenemos las herramientas suficientes para explicarlo.
Basta imaginar que es un grafo formado por una cantidad finita de objetos y morfismos de una categoría.

El concepto de conmutatividad de un diagrama es el habitual de otros campos.
Como ejemplo, la propiedad \eqref{cat:3} se puede expresar diciendo que el diagrama
\[
\begin{tikzcd}
A \arrow[rr,"f"]\arrow[rrdd,pos=0.1,"g \circ f" sloped] \arrow[dd,"h\circ g \circ f" left] & & B \arrow[lldd,pos=0.4,"h\circ g" sloped, crossing over] \arrow[dd,"g"]\\
&\\
C & & D \arrow[ll,"h"]
\end{tikzcd}
\]
conmuta.

Dado un morfismo $f \colon A \to B$, llamaremos dominio de $f$ a $\dom f = A$.
De igual manera, el codominio de $f$ será $\cod f = B$. Además, llamaremos a la colección de morfismos de una categoría $M(\cat)$.

No hay falta de ejemplos de categorías, ni siquiera si nos limitamos a un campo específicio de la matemática.
Algunos ejemplos ilustrativos son los siguientes:

\begin{example}
La categoría $\Set$ formada por los conjuntos como objetos y funciones entre conjuntos como morfismos.
La composición de morfismos se corresponde, como es de esperar, con la composición de funciones.
\end{example}

\begin{example}
La categoría $\Grp$ formada por los grupos como objetos y homomorfismos entre grupos como morfismos.
\end{example}

\begin{example}
La categoría $\Top$ formada por los espacios topológicos como objetos y funciones continuas como morfismos.
\end{example}

\begin{example}
Todo \newterm{conjunto parcialmente ordenado} (o \index{poset|see {conjunto parcialmente ordenado}}\emph{poset}) $(X,\leq)$ puede verse como una categoría, donde los objetos son los elementos de $X$ y hay un único morfismo $f \colon a \to b$ si y sólo si $a \leq b$.
Este es nuestro primer ejemplo donde los morfismos no son equivalentes a función.
La composición de dos morfismos aquí es equivalente a la propiedad transitiva de la relación de orden $\leq$.
Es más, el morfismo identidad es equivalente a la propiedad reflexiva.
\end{example}

También nos interesarán en algunos casos categorías que posean una estructura finita dada, como:
\begin{example}
La categoría $\mathbb{1}$ formada por un sólo objeto y su morfismo identidad:
\[ \begin{tikzcd}
A \arrow[loop above,"\id_A"]
\end{tikzcd}\]
\end{example}

\begin{example}
La categoría $\mathbb{2}$:
\[ \begin{tikzcd}
A \arrow[loop above,"\id_A"] \arrow[r,"f"] & B \arrow[loop above,"\id_B"]
\end{tikzcd}\]
\end{example}

Las categorías finitas serán de gran uso cuando hablemos más adelante de los diagramas.
Por otro lado, las categorías con las que trabajaremos a menudo serán enormes, donde a veces la teoría de conjuntos habitual se nos quedará pequeña.
Por ejemplo, en la teoría de conjuntos habitual, el conjunto de todos los conjuntos es una noción inconsistente como consecuencia del Teorema de Cantor.
Por esta razón, hablaremos a menudo de \textquote{colección} o \textquote{familia} y evitaremos la palabra \textquote{conjunto} en general.

En ocasiones, sí nos será útil que la colección de objetos o morfismos forme un conjunto.
Para ello, introducimos las siguientes definiciones:

\begin{definition}
Una categoría $\cat$ es \index{categoría!pequeña}\emph{pequeña} si $O(\cat)$ y $M(\cat)$ forman conjuntos.
\end{definition}

Por ejemplo, toda categoría finita es pequeña, pero $\Set$ no es pequeña.

\begin{definition}
Una categoría $\cat$ es \index{categoría!localmente pequeña}\emph{localmente pequeña} si para todo par de objetos $A$ y $B$ en $\cat$, $C(A,B)$ forma un conjunto.
\end{definition}

Todos los ejemplos que hemos dado son localmente pequeños.
Para poder dar ejemplos de categorías que no son localmente pequeñas, necesitaremos más conceptos.
Casi siempre trabajaremos con categorías localmente pequeñas.

También podemos construir categorías a partir de otras categorías.
El ejemplo más importante es el de la categoría opuesta:
\begin{definition}
Dada una categoría $\cat$, la \index{categoría!opuesta}\emph{categoría opuesta} o \index{categoría!dual}\emph{dual} $\cat^{op}$ es la categoría con los mismos objetos que $\cat$ y donde un morfismo $f \colon B \to A$ en $\cat^{op}$ es un morfismo $f \colon A \to B$ en $\cat$.
\end{definition}

Evidentemente se tiene que $\left(\cat^{op}\right)^{op} = \cat$.
A menudo nos encontraremos con conceptos que están relacionados por el dual de una categoría.
Por ejemplo, el objeto terminal de una categoría (que veremos más adelante) no es más que el objeto inicial de su categoría opuesta.

\begin{definition}
Sea $A$, $B$ objetos de una categoría $\cat$.
Un morfismo $f \colon B \to C$ de una categoría es un monomorfismo (o mónico) si para todo morfismo $g \colon A \to B$ y $h \colon A \to B$ tales que $f \circ g = f \circ h$, entonces $g = h$.
\end{definition}

\begin{proposition}
En \Set, un morfismo es mónico si y sólo si es inyectivo.
\end{proposition}

\begin{proof}
Sea $f \colon B \to C$ un monomorfismo. Sean $b, b' \in B$ tales que $f(b) = f(b')$. Sea $A = \{b\}$ y definimos $g \colon A \to B$ como $g(b)=b'$.
Entonces $f(\id_A(b)) = f(g(b))$
como $b$ es el único elemento de $A$, tenemos que:
\[ f \circ \id_A = f \circ g \]
luego, como $f$ es monomorfismo:
\[ \id_A = g \]
Entonces $b'=g(b)=\id_A(b)=b$.

Consideramos ahora $f \colon B \to C$ un morfismo inyectivo.
Sean los morfismos $g, h \colon A \to B$ tales que $f \circ g = f \circ h$.
Sea $a \in A$ cualquiera. Como $f(g(a)) = f(h(a))$ y $f$ es inyectiva, entonces $g(a) = h(a)$.
Es decir, se tiene que $g = h$. 
\end{proof}

\begin{definition}
Un morfismo $f \colon A \to B$ es epimorfismo (o épico) si para cualquier par de morfismos $g \colon B \to C$ y $h \colon B \to C$, se tiene que $g \circ f = h \circ f$ implica que $g = h$.
\end{definition}

\begin{proposition}
En \Set, un morfismo es épico si y sólo si es sobreyectivo.
\end{proposition}

\begin{proof}
Sea $f \colon A \to B$ un epimorfismo. Supongamos que $f$ no fuera sobreyectivo. Entonces existe $b \in B$ que no es imagen de ningún elemento de $A$. Sean $g, h \colon B \to \{1,2\}$ tal que $g(x)=h(x)=1$ para todo $x \in B \setminus \{b\}$, $g(b) = 1$ y $h(b) = 2$. Entonces tenemos que $g \circ f = h \circ f$. Como $f$ es epimorfismo, esto implica que $g = h$, pero esto entra en una contradicción.

Sea $f \colon A \to B$ un morfismo sobreyectivo. Sean $g, h \colon B \to C$ tales que $g \circ f = h \circ f$. Para todo $b \in B$, existe $a \in A$ tal que $f(a)=b$, luego: $g(b) = g(f(a)) = h(f(a)) = h(b)$, luego $g = h$.
\end{proof}

\begin{definition}
Un morfismo $f \colon A \to B$ es \newterm{isomorfismo} si existe $g \colon B \to A$ tal que $g \circ f = \id_A$ y $f \circ g = \id_B$.

Dos objetos $A$ y $B$ se dirán \emph{isomorfos} si existe un isomorfismo entre ellos.
\end{definition}

Claramente, en $\Set$, un morfismo mónico y épico es isomórfico.
Sin embargo, todos los morfismos de un poset son épicos y mónicos, pero sólo los morfismos identidad son isomorfismos.

\section{Functores}
Parece que muchos conceptos como $\Set$ o $\Top$ forman categorías.
De aquí surge la siguiente cuestión: ¿Podríamos hacer una categoría de categorías?
Es decir, una categoría donde los objetos mismos sean categorías.
Para ello, debemos definir morfismos entre categorías que respetan su estructura interna.
Aunque la misma razón por la que no hay conjunto de conjuntos, no hay una categoría de categorías, esta cuestión motiva la definición de functor:

\begin{definition}\label{def:functor}
Un \newterm{functor} $F$ entre un par de categorías $\cat$ y $\mathcal{D}$ es un par de funciones:
\begin{itemize}
\item $F_O : O(\cat) \to O(\mathcal{D})$.
\item $F_M : M(\cat) \to M(\mathcal{D})$.
\end{itemize}
tal que:
\begin{itemize}
\item Respeta el dominio y codominio de los morfismos: A cada morfismo $f : A \to B$:
\[ F_M(f) : F_O(A) \to F_O(B) \]
\item Respeta el morfismo identidad:
\[ F_M(\id_A) = \id_{F_O(A)} \]
\item Respeta la composición de morfismo: Para todo morfismos $f : A \to B$, $g : B \to C$.
\[ F_M(g \circ f) = F_M(g) \circ F_M(f) \]
\end{itemize}
\end{definition}

Escribiremos $F \colon \cat \to \mathcal{D}$ y a menudo usaremos $F$ para referirnos a $F_M$ ó $F_O$ según el contexto.

Viendo los functores como funciones entre categorías, es natural que queramos definir composición entre functores.
\begin{definition}
Dado dos functores $F \colon \cat \to \mathcal{D}$ y $G \colon \mathcal{D} \to \mathcal{E}$, definimos $G \circ F$ como el functor con:
\begin{itemize}
  \item $(G \circ F)_O = G_O \circ F_O$.
  \item $(G \circ F)_M = G_M \circ F_M$.
\end{itemize}
\end{definition}
\subsection{Functores en Haskell}
Recordemos que en el contexto de la programación con tipos, Haskell sólo trabaja con tipos de la categoría \code{Hask}.
Los functores con los que podemos trabajar son de endofunctores de \code{Hask}, es decir, functores que van de \code{Hask} a \code{Hask} -- de tipos a tipos.

Haskell implementa los functores a través de sus \code{typeclasses}.
\begin{minted}{haskell}
class Functor f where
  fmap :: (a -> b) -> f a -> f b
\end{minted}

La \code{f} actúa como $F$ y como $F_O$ en \ref{def:functor}.
Si hay dudas de qué es lo que hace \code{fmap} aquí, resulta útil añadir unos paréntesis en el lugar adecuado:

\begin{minted}{haskell}
fmap :: (a -> b) -> (f a -> f b)
\end{minted}

\code{fmap} toma un función cualquiera \code{a -> b} y la transforma en una función del tipo \code{f a -> f b}.
Esto es precisamente lo que antes llamábamos $F_M$.
Con esto en mente, adaptemos las leyes de los functores en Haskell:
\begin{minted}{haskell}
fmap id = id
fmap (g . f) = fmap g . fmap f
\end{minted}
Recordemos que en Haskell, la composición de funciones $g \circ f$ se expresa usando \code{g . f}.
Antes de pasar a unos ejemplos, hay que tener en cuenta que Haskell no dispone de ningún mecanismo que comprueba que estas igualdades se cumplen.
Queda como responsabilidad del programador que la clase \code{Functor} no se aplique sobre tipos que no cumplen las leyes de functores. 

\begin{example}
Nuestro primer ejemplo es \code{Maybe}.
\begin{minted}{haskell}
data Maybe a = Nothing | Just a 
\end{minted}

Podemos leer esta definición como \code{Maybe a} contiene o bien un valor constante o bien un valor del tipo \code{a}.
Hay que tener cuidado con la diferencia entre \code{Maybe} y \code{Just}.
\code{Maybe} es lo que llamamos un constructor de tipo.
Veremos que este constructor de tipo, que asocia a cada tipo \code{a} el tipo \code{Maybe a}, es un functor.
Por otro lado, \code{Just} es una función polimórfica --y por lo tanto una familia de morfismos en \code{Hask}-- de \code{a} a \code{Maybe a}.

Para ver que \code{Maybe} es un functor, tenemos que describir como actúa sobre morfismos, ahí es donde entra en juego la función \code{fmap}.

\begin{minted}{haskell}
instance Functor Maybe where
  fmap f Nothing = Nothing
  fmap f (Just x) = Just (f x) 
\end{minted}
\end{example}

\subsection{Functores especiales}
Consideramos ahora el concepto de functor contravariante a través de un cambio en la definición de functor:
\begin{definition}\label{def:cofunctor}
Un \index{functor!contravariante}\emph{functor contravariante} $F$ entre un par de categorias $\cat$ y $\mathcal{D}$ es un par de funciones:
\begin{itemize}
\item $F_O : O(\cat) \to O(\mathcal{D})$.
\item $F_M : M(\cat) \to M(\mathcal{D})$.
\end{itemize}
tal que:
\begin{itemize}
\item \textbf{Intercambia} el dominio y codominio de los morfismos: A cada morfismo $f : A \to B$:
\[ F_M(f) : F_O(B) \to F_O(A) \]
\item Respeta el morfismo identidad:
\[ F_M(\id_A) = \id_{F_O(A)} \]
\item \textbf{Intercambia} la composición de morfismo: Para todo morfismos $f : A \to B$, $g : B \to C$.
\[ F_M(g \circ f) = F_M(f) \circ F_M(g) \]
\end{itemize}
\end{definition}

Los functores con los que hemos trabajado hasta ahora serán llamados \index{functor!covariante}\emph{functores covariantes}, pero a menudo prescindiremos del adjetivo.
Una razón para esto es que no hay diferencia esencial entre functor contravariante y functor covariante cuando observamos que los \textquote{intercambios} en la definición son equivalentes a que el functor parta de $\cat^{op}$.
Por esta razón, hablaremos de functor contravariante cuando tratemos con un functor covariante de la forma:
\[ F \colon \cat^{op} \to D \]

Otros functores reciben nombres especiales por su dominio o su codominio:
\begin{definition}
Un functor $F \colon \cat \to \cat$ se denomina \newterm{endofunctor}.
\end{definition}
%Categories and Haskell 1
\begin{example}
Un ejemplo sencillo es el \newterm{endofunctor potencia} $\mathcal{P} \colon \Set \to \Set$ que envía un conjunto $A$ a su conjunto potencia $\mathcal{P}(A)$ y envía a las funciones entre conjuntos $f \colon A \to B$ a la función $\mathcal{P} f \colon \mathcal{P}(A) \to \mathcal{P}(B)$ definida como:
\[ \mathcal{P} f (S) = \{f(s) \colon s \in S\}, \quad S \subseteq A \]
\end{example}

\begin{definition}
Un functor $F \colon \cat^{op} \times \cat \to \mathcal{D}$ se denomina \newterm{bifunctor}.
\end{definition}
Aquí $\cat^{op} \times \cat$ es la categoría producto de $\cat^{op}$ y $\cat$, definida más adelante en la sección 1.5.

\begin{definition}
Un functor $F \colon \cat^{op} \to \Set$ se denomina \newterm{prehaz}.
\end{definition}

\section{Construcciones universales}
Podemos caracterizar ciertos objetos en una categoría por alguna propiedad especial que cumplan. En lugar de pedir que sólo haya un objeto que cumpla dicha propiedad, nos limitamos a pedir que todos los objetos que cumplan la propiedad sean isomorfos. Estas propiedades son llamadas \emph{propiedades universales}.

\begin{definition}
Decimos que un objeto $A \in \cat$ es \newterm{inicial} si para cada objeto $B \in \cat$, hay exactamente un morfismo $f \colon A \to B$.

Análogamente, decimos que un objeto $B \in \cat$ es \newterm{terminal} si para cada objeto $A \in \cat$, hay exactamente un morfismo $f \colon A \to B$.
\end{definition}

Cuando hablemos de un objeto inicial, usaremos a menudo la notación $0$.
Análogamente, usaremos $1$ para referirnos a un objeto terminal.

\begin{example}
Veamos algunos ejemplos en categorías usuales:
\begin{itemize}
  \item En $\Set$, el conjunto vacío es un objeto inicial y cualquier conjunto unitario es terminal.
  \item En $\Grp$, el grupo trivial es inicial y terminal.
  \item En $\Ring$, el anillo unitario es terminal y el anillo $(\Z,+,\cdot)$ es inicial.
  \item En un poset, visto como categoría, el objeto inicial es el elemento mínimo global y el objeto terminal es el elemento máximo global, si existen.
  \item En $\Hask$, el objeto terminal se define como el tipo de dato con un sólo valor posible.
  Se suele usar \code{()}, que puede entenderse como una $0$-tupla.
  El objeto inicial se define como el tipo de dato sin constructor:
  \begin{minted}{haskell}
  data Empty
  \end{minted}
\end{itemize}
\end{example}

De estos ejemplos vemos que:
\begin{itemize}
  \item Los objetos iniciales y terminales no son necesariamente únicos.
  \item Un objeto puede ser inicial y terminal.
  En dicho caso, el objeto en cuestión recibe el nombre de \index{objeto!cero}\emph{objeto cero}.
  \item Una categoría puede no poseer objetos iniciales o terminales.
\end{itemize}

Aunque no haya unicidad de objetos iniciales o terminales, sí hay una relación entre todos los objetos iniciales o terminales.
\begin{proposition}
Sea $\cat$ una categoría. Sean $A$ y $B$ objetos iniciales (o terminales) de $\cat$.
Entonces existe un único isomorfismo $A \to B$.
\end{proposition}
\begin{proof}
Como $A$ es inicial, existe un único morfismo $f \colon A \to B$.
Como $B$ es inicial, existe un único morfismo $g \colon B \to A$.
Entonces $g \circ f \colon A \to A$.
Al ser $A$ inicial, sólo hay un morfismo $A \to A$, que debe ser el morfismo identidad, luego $g \circ f = \id_A$.
Análogamente, $f \circ g = \id_B$.

Luego $f$ es isomorfismo.
\end{proof}

\section{Diagramas y conos}
En estas páginas hemos trabajado a menudo con categorías con infinitos objetos e innumerables morfismos.
Sin embargo, resulta de interés estudiar una estructura particular de objetos y morfismos que podamos encontrar dentro de esa categoría.
Para ello, formalizaremos un concepto que hemos manejado antes, los diagramas.
\begin{definition}
Sean $\cat$ y $\mathcal{I}$ dos categorías. Un \newterm{diagrama} de $\cat$ con \emph{forma} $\mathcal{I}$ es un functor $D \colon \mathcal{I} \to \cat$.
\end{definition}
Es decir, técnicamente diagrama será sólo otra palabra para functor.
En la práctica, usaremos diagrama cuando querramos distinguir una estructura particular dentro de una categoría.
También se llamará a $\mathcal{I}$ \index{categoría!índice}\emph{categoría índice} del diagrama.

\begin{example}\label{ejemplo-span}
Sea $\mathcal{I}$ la siguiente categoría índice:
\[ \begin{tikzcd}
A & B \arrow[l] \arrow[r] & C
\end{tikzcd} \]
Entonces el diagrama $D \colon \mathcal{I} \to \cat$ es llamado \newterm{span}.
\end{example}

\begin{definition}
Sea $\cat$ una categoría y $D \colon \mathcal{I} \to \cat$ un diagrama de forma $\mathcal{I}$.
Un \newterm{cono} en el diagrama $D$ es un objeto $V \in \cat$ (que llamamos \newterm{vértice}) y una familia de morfismos $f_A \colon C \to D A$ para todo $A \in \mathcal{I}$, tal que para todo morfismo $g \colon A \to B$ de $\mathcal{I}$, el siguiente diagrama conmuta:
\[ \begin{tikzcd}
 & D A \arrow[dd,"D g"]\\
V \arrow[ur,"f_A"] \arrow[dr,"f_B"]\\
 & D B
\end{tikzcd} \]
\end{definition}

Una buena técnica para estudiar una nueva construcción es preguntarse: ¿Podemos formar una categoría con ella?

Lo que nos queda por hacer para crear una categoría de conos de un diagrama fijo $D$ es definir un morfismo entre conos.
Resulta intuitivo definirlo a partir de un morfismo entre los vértices.
Si $C$ y $C'$ son dos conos con vértices $V$ y $V'$ respectivamente, definimos el morfismo $C \to C'$ como un morfismo $u \colon V \to V'$ tal que el siguiente diagrama conmuta para todo $A \in \mathcal{I}$:
\[ \begin{tikzcd}
V \arrow[dr,"f_A"] \arrow[dd,"u"] &\\
 & D A\\
V' \arrow[ur,"f'_A"]
\end{tikzcd} \]

\begin{proposition}
Dado una categoría $\cat$ y un diagram $D \colon \mathcal{I} \to \cat$, los conos de $D$ forma una categoría llamada $\text{Cono}(D)$ con los morfismos descritos anteriormente.
\end{proposition}
La demostración es directa gracias a que $\cat$ es una categoría.

En esta categoría de conos, resulta de especial interés el objeto terminal.

\begin{definition}
Dado una categoría $\cat$ y un diagrama $D \colon \mathcal{I} \to \cat$, el \newterm{límite del diagrama} $D$ es el objeto terminal de $\text{Cono}(D)$.
\end{definition}

Usaremos la notación $\underset{\longleftarrow}\lim D$ para referirnos al límite de un diagrama si existe.

Podemos considerar el dual de estos conceptos.

\begin{definition}
Un \newterm{cocono} de $\cat$ es un cono de $\cat^{op}$.
En otras palabras, un cocono en un diagrama $D \colon \mathcal{I} \to \cat$ es un objeto $V \in \cat$ y una familia de morfismos $f_A \colon D A \to V$ para todo $A \in \mathcal{I}$, tal que para todo morfismo $g \colon B \to A$ de $\mathcal{I}$, el siguiente diagrama conmuta:
\[ \begin{tikzcd}
 & D A \arrow[dl,"f_A"]\\
V\\
 & D B \arrow[ul,"f_B"] \arrow[uu,"D g"]
\end{tikzcd} \]
\end{definition}

\begin{definition}
Un \newterm{colímite} es el objeto terminal de la categoría de coconos de un diagrama. Equivalentemente es el objeto inicial de la categoría de conos.
\end{definition}

\begin{definition}
Una categoría $\cat$ es \index{categoría!completa}\index{categoría!cocompleta}\emph{(co)completa} si existen todos los (co)límites de diagramas $D \colon \mathcal{I} \to \cat$ con $\mathcal{I}$ categoría pequeña.

Una categoría es \index{categoría!bicompleta}\emph{bicompleta} si es completa y cocompleta.
\end{definition}

\section{Productos y coproductos}

Resulta importante familiarizarse con la idea de límite antes de continuar, porque nos resultará muy útil para definir nuevos conceptos.

\begin{definition}
Sea $\mathcal{I}$ la categoría de dos objetos $I_1$ y $I_2$ y sin morfismos aparte de los morfismos identidad. El \newterm{producto} de dos objetos $A$ y $B$ en $\cat$ es el límite del diagrama $D \colon \mathcal{I} \to \cat$ con $D(I_1) = A$ y $D(I_2) = B$.

Escribiremos $A \times B$ para denotar el producto de $A$ y $B$.
\end{definition}

Desenvolvamos las definiciones para entender qué significa realmente el producto de dos objetos $A$ y $B$.
Primero obsérvese que el diagrama $D$ actúa para \textquote{seleccionar} los objetos $A$ y $B$ y sus morfismos identidad.
Puede que haya algún morfismo entre $A$ y $B$, pero no son relevantes para el diagrama y, en consecuencia, ni para el límite.

Un cono sobre $D$ es de la forma:
\begin{equation}\label{producto-cono} \begin{tikzcd}
 & A\\
V \arrow[ur,"f_A"] \arrow[dr,"f_B"]\\
 & B
\end{tikzcd} \end{equation}

El producto $A \times B$, límite de $D$, puede ser visto como el cono terminal en la categorías de conos sobre $D$.
Es decir, para todo cono como en \ref{producto-cono}, debe existir un único morfismo de dicho cono al $A \times B$.
Desenvolvamos las definiciones aún más.
Digamos que el producto está formado por los objetos $A \times B$ (abusando un poco de notación aquí), $p_A \colon A \times B \to A$ y $p_B \colon A \times B \to B$.
Para todo $V$, $f_A \colon V \to A$ y $f_B \colon V \to B$, debe existir un $h \colon V \to A \times B$ tal que el siguiente diagrama conmute:

\[\begin{tikzcd}
 & A\\
V \arrow[ur,"f_A"] \arrow[dr,"f_B"] \arrow[r,dashed,"h"] & A \times B \arrow[u,"p_A"] \arrow[d,"p_B"]\\
 & B
\end{tikzcd}\]

o algebraicamente:
\[ f_A = p_A \circ h \]
\[ f_B = p_B \circ h \]
Es decir, dar un par de morfismos $V \to A$ y $V \to B$ es equivalente a dar un morfismo $V \to A \times B$.
Luego, podemos volver a los objetos $A$ y $B$ usando unos morfismos fijos $p_A$ y $p_B$ (que llamamos \index{morfismo!proyección}\emph{proyecciones}).

\begin{definition}
Sea $\mathcal{I}$ la categoría de dos objetos $I_1$ y $I_2$ y sin morfismos aparte de los morfismos identidad. El \newterm{coproducto} de dos objetos $A$ y $B$ en $\cat$ es el colímite del diagrama $D \colon \mathcal{I} \to \cat$ con $D(I_1) = A$ y $D(I_2) = B$.

Escribiremos $A + B$ para denotar el coproducto de $A$ y $B$.
\end{definition}

Desarrollando como antes la definiciones tenemos que para todo $V$, $f_A \colon A \to V$, $f_B \colon B \to V$, se tiene que existe $h \colon A + B \to V$ tal que:

\[\begin{tikzcd}
 & A \arrow[dl,"f_A"] \arrow[d,"i_A"]\\
V & A + B \arrow[l,dashed,"h"]\\
 & B \arrow[ul,"f_B"] \arrow[u,"i_B"]
\end{tikzcd}\]
donde $i_A$ e $i_B$ son llamadas \newterm{inyecciones naturales}.

\begin{example}
No es de extrañar que en $\Set$, el producto de dos objetos es el producto cartesiano de dos conjuntos.
Por otro lado, el coproducto de dos conjuntos se corresponde con la unión disjunta.
\end{example}

\begin{example}\label{ejemplo-producto-pos}
Consideramos la categoría $Pos$ de conjuntos parcialmente ordenados (posets) donde los morfismos son funciones monótonas.
El producto de dos posets $P$ y $Q$ es un poset $P \times Q$ de pares $(p,q)$ con $p \in P$ y $q \in Q$ ordenado parcialmente por:
\[ (p,q) \leq (p',q') \sii p \leq p' \text{ y }q \leq q' \]
Las proyecciones $p_1 \colon P \times Q \to P$ y $p_2 \colon P \times Q \to Q$ son evidentemente monótonas, luego son morfismos de $Pos$.
\end{example}

\begin{example}
En $\Hask$, el producto de dos tipos \code{a} y \code{b} es el par \code{(a,b)} y las proyecciones son las funciones \code{fst :: (a,b) -> a} y \code{snd :: (a,b) -> b}.
También tenemos que \code{(a,b)} es isomorfo a:
\begin{minted}{haskell}
newtype Pair a b = Pair a b
\end{minted}
A su vez, el coproducto de dos tipos \code{a} y \code{b} es el tipo \code{Either a b}, cuya definición es:
\begin{minted}{haskell}
data Either a b = Left a | Right b
\end{minted}
Es más, tenemos que todos los tipos de datos compuestos son isomorfos a productos y coproductos.
\end{example}

%Riehl 1
\begin{example}
En \index{categoría!{de categorías pequeñas}}\emph{categoría de categorías pequeñas} $\Cat$, el producto de dos categorías $\cat$ y $\mathcal{D}$ es una categoría $\cat \times \mathcal{D}$ cuyos objetos son pares ordenados $(A,B)$ con $A \in \cat$ y $B \in \mathcal{D}$.
Los morfismos, composición e identidad se definen a manera análoga por componentes.
\end{example}

Podemos definir producto $n$-ario de objetos como el límite de un diagrama con una categoría índice discreta (es decir, sin morfismos a parte de los morfismos identidad) de $n$ objetos.
\begin{remark}\mbox{}
\begin{itemize}
  \item Si existen productos binarios, entonces existe cualquier produco $n$-ario con $n \geq 2$.
  \item En el caso particular de $n=1$, el $1$-producto de cualquier objeto es precisamente el mismo objeto.
  \item En el caso particular de $n=0$, el $0$-producto es precisamente el objeto terminal.
\end{itemize}
\end{remark}

Por esta razón, cuando una categoría tiene productos binarios y objeto terminal, se dice que tiene \emph{todos los productos finitos}.

Hemos visto el producto y el coproducto como casos particulares de límites.
Otros límites importantes son:
\begin{definition}
Dado un diagrama \newterm{span}
\[ \begin{tikzcd}
A & B \arrow[l] \arrow[r] & C
\end{tikzcd} \]
sobre una categoría, su colímite se denomina \newterm{pushout}.
\end{definition}

Un uso típico de pushouts está presente en topología:
\begin{example} % https://ncatlab.org/nlab/show/limits+and+colimits+by+example#pullback
Dado un span
\[ \begin{tikzcd}
X & A \arrow[l,"f" above] \arrow[r,"i"] & Y
\end{tikzcd} \]
en $\Top$, donde $i \colon A \to Y$ es una inclusión, entonces el pushout se llama \newterm{espacio de adjunción} y se escribe $X \cup_f Y$.
El espacio de adjunción se corresponde con el espacio cociente de la unión disjunta de $X$ e $Y$ bajo la relación de equivalencia generada por $x \sim f(x)$. 
\end{example}

\begin{definition}
Dado un diagrama \newterm{cospan} (dual de span)
\[ \begin{tikzcd}
A \arrow[r] & B & C \arrow[l]
\end{tikzcd} \]
sobre una categoría, su límite se denomina \newterm{pullback}.
\end{definition}
\begin{definition}
Dado un diagrama de \newterm{pareja paralela}
\[ \begin{tikzcd}
A \arrow[r,bend left] \arrow[r,bend right] & B
\end{tikzcd} \]
sobre una categoría, su límite se denomina \newterm{ecualizador} y su colímite, \newterm{coequalizador}.
\end{definition}

En resumen:

\begin{center}
\begin{tabular}{ c | c | c }
  Forma & Límite & Colímite\\
  \hline
  $\emptyset$ & Objeto terminal & Objeto inicial \\
  Discreta finita & Producto & Coproducto \\
  Span & - & Pushout \\
  Cospan & Pullback & - \\
  Pareja paralela & Ecualizador & Coecualizador
\end{tabular}
\end{center}

\section{Functores continuos}
Claramente un functor envía conos a conos, pero no es cierto que envíe límites a límites en general.

\begin{definition}
Un functor $F \colon \cat \to \mathcal{D}$ se dice que \emph{preserva límites de forma} $\mathcal{I}$ si para todo diagrama $D \colon \mathcal{I} \to \cat$ con cono límite $L = \underset{\longleftarrow}\lim D$, el cono $F L$ es el cono límite del diagrama $F \circ D$.
Se suele expresar como:
\[ F\left(\underset{\longleftarrow}\lim D\right) = \underset{\longleftarrow}\lim \left(F \circ D\right)\]
Análogamente, $F$ se dice que \emph{preserva colímites de forma} $D$ si:
\[ F\left(\underset{\longrightarrow}\lim D\right) = \underset{\longrightarrow}\lim \left(F \circ D\right)\]
\end{definition}

La uso de límites aquí nos recuerda a la definición de funciones continuas de la matemática clásica.
Esto lleva a la definición de functor continuo y cocontinuo:
\begin{definition}
Un functor es \index{functor!(co)continuo}\emph{functor (co)continuo} si preserva (co)límites de cualquier forma pequeña.
\end{definition}
%Mac Lane
%\begin{theorem}
%Para una categoria localmente pequeña $\cat$ y un objeto $A \in \cat$, el functor $\cat(A,-)$ es continua.
%\end{theorem}
%\begin{proof}
%Sea $D \colon \mathcal{I} \to \cat$ un diagrama pequeño cualquiera con cono límite $\underset{\longleftarrow}\lim D$ formado por un vértice $V$ y morfismos 
%\end{proof}

En algunos contextos, resulta más útil debilitar la definición de functor continuo a diagramas de forma $\omega = (\mathbb{N}, \leq)$, esto nos lleva a la definición de functor $\omega$-continuo.
Un diagrama de forma $\omega$ se refiere a una cada de morfismos como:
\[ \dots \rightarrow A_3 \rightarrow A_2 \rightarrow A_1 \Rightarrow A_0 \]
En estos casos, se suele identificar el diagrama con la sucesión $\{A_n\}_{n \in \mathbb{N}}$, de manera que el límite se escribe:
\[ \underset{\longleftarrow}\lim A_i\]
\begin{definition}
Un functor es $\omega$\emph{-(co)continuo} si preserva (co)límites de forma $\omega$.
\end{definition}
\begin{example}
Dado un primo $p$, los enteros $p$-ádicos se definen como el límite del diagrama:
\[ \dots \rightarrow \mathbb{Z}/p^3\Z \rightarrow \mathbb{Z}/p^2\Z \rightarrow \mathbb{Z}/p\Z \]
en la categoría de anillos.
\end{example}

\section{Transformaciones naturales}
\begin{definition}
Una transformación natural $\mu$ entre dos functores $F, G \colon \cat \to \mathcal{D}$ es una colección de morfismos
\[ \mu_A \colon F A \to G A \]
donde $A \in \cat$ tal que para todo $f \colon A \to B$ el diagrama:
\[
\begin{tikzcd}
F A \arrow[r,"\mu_A"] \arrow[d,"F f" left] & G A \arrow[d,"G f"]\\
F B \arrow[r,"\mu_B"] & G B
\end{tikzcd}
\]
conmuta.
\end{definition}

Usaremos la notación $\mu \colon F \Rightarrow G$ para una transformación natural $\mu$.
Llamamos $\mu_A$ \newterm{componente} de $\mu$ en $A$.
La conmutatividad del diagrama se denomina \emph{propiedad de naturalidad} y es equivalente a:
\[ G f \circ \mu_A = \mu_B \circ F f \]
para todo morfismo $f \colon A \to B$ en $\cat$.

Hay dos formas de componer transformaciones naturales.
\begin{definition}
Dado dos transformaciones naturales $\mu \colon F \Rightarrow G$ y $\eta \colon G \Rightarrow H$ donde $F,G,H \colon \cat \to \mathcal{D}$ son functores, su composición horizontal $\eta \circ \mu \colon F \Rightarrow H$ viene dada por
\[ (\eta \circ \mu)_A = \eta_A \circ \mu_A \]
para todo objeto $A \in \cat$.
\end{definition}
\[ \begin{tikzcd}{\cat} \arrow[r,"F", bend left=100, ""{name=U}] \arrow[r,"G", pos=0.2,""{name=M, pos=0.5,below},""{name=M', pos=0.5,above}] \arrow[r,"H" below, bend right=100, ""{name=D, below}] & \mathcal{D}\\
\arrow[Rightarrow, from=U, to=M, "\mu"] \arrow[Rightarrow, from=M', to=D, "\eta"]\end{tikzcd}\]

\begin{lemma}\label{functor-categoria}
Dados dos categorías $\cat$ y $\mathcal{D}$, existe una categoría donde los objetos son functores $\cat \to \mathcal{D}$ y los morfismos son transformaciones naturales.

Denotamos dicha categoría como $\mathcal{D}^\cat$ ó $[\cat,\mathcal{D}]$.
\end{lemma}
\begin{proof}
Vemos primero que podemos definir una transformación natural identidad $\iota \colon F \Rightarrow F$ dado por $\iota_A = \id_{F A}$ para todo $A \in \cat$.
Por otro lado, que la composición de transformaciones naturales es asociativa es trivial.
\end{proof}

% Milewski
Nos interesa ahora definir una composición de transformaciones naturales de la forma:
\begin{equation}\label{ver-comp} \begin{tikzcd}{\cat} \arrow[r,"F", bend left, ""{name=FU}] \arrow[r,"G" below, bend right, ""{name=FD, below}] & \mathcal{D} \arrow[r,"H",bend left, ""{name=GU}] \arrow[r,"K" below, bend right, ""{name=GD, below}] & \mathcal{E}
\arrow[Rightarrow, from=FU, to=FD, "\mu"] \arrow[Rightarrow, from=GU, to=GD, "\eta"]\end{tikzcd}\end{equation}
Nuestro objeto será definir una transformación natural que vaya de $H \circ F$ a $G \circ K$.
%Riehl/Peter Selinger: Introduction to categorical logic. pdf, page 41
Para ello necesitaremos describir dos formas en las que se pueden componer un functor y una transformación natural:
\begin{definition}
Si $F \colon \cat \to \mathcal{D}$ y $G, H\colon \mathcal{D} \to \mathcal{E}$ son functores y $\mu \colon G \Rightarrow H$ es una transformación natural, el \newterm{whiskering izquierdo}:
\[ \mu \circ F \colon G \circ F \Rightarrow H \circ F \]
está definido como:
\[ (\mu \circ F)_A = \mu_{F A} \]
\end{definition}
La naturalidad de $(\mu \circ F)$ es consecuencia de la naturalidad de $\mu$.

Se suele eliminar $\circ$ y escribir sencillamente $\mu F$.

Por otro lado:
\begin{definition}
Si $F, G \colon \cat \to \mathcal{D}$ y $H\colon \mathcal{D} \to \mathcal{E}$ son functores y $\mu \colon F \Rightarrow G$ es una transformación natural, el \newterm{whiskering derecho}:
\[ H \circ \mu  \colon H \circ F \Rightarrow H \circ G \]
está definido como:
\[ (H \circ \mu)_A = H(\mu_A) = H\mu_A \]
\end{definition}
Aquí la naturalidad no es tan evidente.
\begin{proposition}
El \emph{whiskering} derecho es una transformación natural.
\end{proposition}
\begin{proof}
Por la naturalidad de $\mu$, se da la igualdad
\[ G f \circ \mu_A = \mu_B \circ F f \]
aplicando $H$ y su propiedad de functorialidad:
\[ H(G f) \circ H\mu_A = H\mu_B \circ H(F f) \]
Lo que prueba la naturalidad de $H \circ \mu$.
\end{proof}
Se suele escribir $H\mu$ en lugar de $H \circ \mu$.

Con esto, ya podemos definir la composición horizontal o composición de Godement.
\begin{definition}
Sean $\cat$, $\mathcal{D}$ y $\mathcal{E}$ tres categorías.
Consideramos los functores $F$, $G$, $H$ y $K$, y las transformaciones naturales $\mu$ y $\eta$ en la configuración dada en \ref{ver-comp}.
Definimos la composición horizontal $\eta * \mu \colon H \circ F \Rightarrow K \circ G$ como:
\[ (\eta * \mu)_A = \eta_{G A} \circ H\mu_A \]
\end{definition}
\begin{proposition}
La composición de horizontal $\eta * \mu$ es una transformación natural.
\end{proposition}
%Riehl
\begin{proof}
Por la naturalidad de $\mu$ y $\eta$ se tiene que cada cuadrado del siguiente diagrama es conmutativo.
\[\begin{tikzcd}
{H(F A)} \arrow[r,"H\mu_A"] \arrow[d,"H(F f)"] & {H(G A)} \arrow[r,"\eta_{G A}"] \arrow[d,"H(G f)"] & {K(G A)} \arrow[d,"K(G f)"]\\
{H(F B)} \arrow[r,"H\mu_A"] & {H(G B)} \arrow[r,"\eta_{G B}"] & {K(G B)}
\end{tikzcd}\]
Luego, componiendo horizontalmente, tenemos que es conmutativo el diagrama:
\[\begin{tikzcd}
{H(F A)} \arrow[rr,"\eta_{G A} \circ H\mu_A"] \arrow[d,"H(F f)"] & & {K(G A)} \arrow[d,"K(G f)"]\\
{H(F B)} \arrow[rr,"\eta_{G B} \circ H\mu_A"] & & {K(G B)}
\end{tikzcd}\]
Esto demuestra la naturalidad de $\eta * \mu$.
\end{proof}

Véase que la composición horizontal se puede ver como una composición vertical tras aplicar un \emph{whiskering} apropiado a cada transformación natural.

\begin{proposition}
Para una transformación natural $\mu \colon F \Rightarrow G$ y una transformación natural identidad $\iota \colon H \Rightarrow H$:
\[ \iota * \mu = H\mu \]
\[ \mu * \iota = \mu H\]
\end{proposition}
\begin{proof}
\begin{align*}
(\iota * \mu)_A & = \iota_{G A} \circ F\mu_A\\
& = \id_{H (G A)} \circ H\mu_A\\
& = H\mu_A
\end{align*}
Por otro lado:
\begin{align*}
(\mu * \iota)_A & = \mu_{H A} \circ G\iota_A\\
& = \mu_{H A} \circ G\id_{H A}\\
& = \mu_{H A} \circ \id_{G(H A)}\\
& = \mu_{H A}\\
& = (\mu H)_A
\end{align*}
\end{proof}

\section{Equivalencias y functores adjuntos}\label{functores-adjuntos}
Obsérvese que tenemos una noción de ``equivalencia'' cuando hablamos de objetos en una categoría que no es estrictamente igualdad, sino isomorfía.
Que dos objetos sean isomorfos significa que existan morfismos entre ellos inversos el uno del otro.
Al nivel de la categoría de categorías pequeñas $\Cat$, podríamos hablar de categorías $\cat$ y $\mathcal{D}$ isomorfas si tienen functores $F \colon \cat \to \mathcal{D}$ y $G \colon \mathcal{D} \to \cat$ tal que $G \circ F = \Id_\cat$ y $F \circ G = \Id_\mathcal{D}$.
Sin embargo, esta idea resulta muy estricta para ser aplicada, en lugar de ello hacemos uso de equivalencias.
Para relajar esta definición, usaremos transformaciones naturales.

\begin{definition}
Una \newterm{equivalencia} entre dos categorías $\cat$ y $\mathcal{D}$ es un par de functores $F \colon \cat \to \mathcal{D}$ y $G \colon \mathcal{D} \to \cat$ de manera que existen dos isomorfismos naturales $G \circ F \cong \Id_\cat$ y $F \cong G \cong \Id_\mathcal{D}$.
\end{definition}

Aunque las equivalencias resultan útiles en la teoría de categorías de orden superior, no sirve para un tipo de functor muy habitual en diversos campos de la matemáticas.

Aquí entra la idea de adjunción.
Antes de entrar formalmente en ellos, veamos un ejemplo que motiva la noción que formalizaremos.

%\begin{example}
%Consideremos el caso de dos conjuntos pre-ordenados $\cat$ y $\mathcal{D}$.
%Podemos verlos como categorías, donde los objetos son los elementos del conjunto y hay un morfismo $A \to B$ si $A \leq B$.
%Una función $F \colon \cat \to \mathcal{D}$ que mantiene el orden puede ser visto sencillamente como un functor sobre estas categorías.

%Un functor $G \colon \mathcal{D} \to \cat$ es un \newterm{adjunción derecha} a $F$ si, para todo $A \in \cat$ y $B \in \mathcal{B}$:
%\[ F A \leq B \sii A \leq G B \]
%\end{example}

Sea $U \colon \Grp \to \Set$ el functor que asocia a cada grupo $(G,*)$ el conjunto de sus elementos $G$.
Esta clase de functor, que sólo ``olvida'' la estructura sobre un objeto, se denomina \index{functor!olvidadizo}\emph{functor olvidadizo}.
Sea $F \colon \Set \to \Grp$ el functor que asocia a cada conjunto su correspondiente \newterm{grupo libre}.
En concreto, para un conjunto $S$, $F S$ puede verse como el grupo formado por la concatenación de palabras cuyas letras son elementos de $S$ y sus correspondiente letras inversas.

Estos functores no son inversos el uno del otro, ni siquiera forman una equivalencia, pero representan una relajación de inversión.
Esta idea se formaliza en la adjunción:

\begin{definition}
Una \newterm{adjunción} entre categorías $\cat$ y $\mathcal{D}$ son functores
\[ F \colon \cat \to \mathcal{D} \qquad U \colon \mathcal{D} \to \cat \]
y dos transformaciones naturales:
\[ \eta \colon \Id_\cat \Rightarrow U \circ F \qquad \varepsilon \colon F \circ U \Rightarrow \Id_\mathcal{D} \]
que satisfagan las identidades:

\[
\begin{tikzcd}
F \arrow[r,Rightarrow,"F \eta"] \arrow[dr,Rightarrow,"\iota_F"] & F \circ U \circ F \arrow[d,Rightarrow,"\varepsilon F"]\\
& F
\end{tikzcd}
\qquad
\begin{tikzcd}
U \arrow[r,Rightarrow,"\eta U"] \arrow[dr,Rightarrow,"\iota_U"] & U \circ F \circ U \arrow[d,Rightarrow,"U \varepsilon"]\\
& U
\end{tikzcd}
\]
donde $\iota_F \colon F \Rightarrow F$ es la transformación natural identidad.
\end{definition}

Dicha adjunción se suele expresar como:
\[ \cat
    \underoverset
      {\underset{F}{\longrightarrow}}{\overset{U}{\longleftarrow}}{\top}
  \mathcal{D}\]
o sencillamente $F \dashv U$.
Se suele decir que $U$ es la adjunta derecha de $F$ y $F$ es la adjunta izquierda de $U$.
Las transformaciones naturales $\eta$ y $\varepsilon$ se llaman \newterm{unidad} y \newterm{counidad} respectivamente.

Volviendo a nuestro ejemplo:
\[ \Set
    \underoverset
      {\underset{F}{\longrightarrow}}{\overset{U}{\longleftarrow}}{}
  \Grp\]
para formar una adjunción basta tomar las transformaciones naturales $\eta_S \colon S \to U (F S)$ que envía cada elemento $s \in S$ a $s$ como elemento de $F S$ y $\varepsilon_G \colon F (U G) \to G$ que envía cada palabra cuyas letras son elementos $G$ a su producto.
Con esto es una simple tarea comprobar que $F \dashv U$.

\begin{proposition}
Sea $F \colon \cat \to \mathbb{1}$ un functor a la categoría de un objeto $\mathbb{1}$ y $A$ un objeto de $\cat$.
Dado $U \colon \mathbb{1} \to \cat$ cuya imagen es $A$, entonces $1$ es objeto terminal de $\cat$ si y sólo si $F \dashv U$.
\end{proposition}
\begin{proof}
Denominemos $*$ el único objeto de $\mathbb{1}$.
Obsérvese primero que $F (U *) = *$ y que $U (F B) = A$ para todo $B \in \cat$. Por lo tanto:
\[ \eta_B \colon B \to A \]
\[ \varepsilon_* = \id_* \]
Entonces, tenemos:
\[ F\eta_B = \id_* \]
\[ \varepsilon_{F B} = \id_1\]
\[ U\varepsilon_* \colon \id_A \]
\[ \eta_{U *} = \eta_A \colon A \to A \]

Luego se da siempre una de las dos igualdades triangulares:
\[ F\eta \circ \varepsilon F = \iota \]

Supongamos que $A$ es un objeto terminal, entonces $\eta$ es natural y $\eta_A = \id_A$, luego se da la otra igualdad triangular y se demuestra que $F \dashv U$.

Supongamos que $F \dashv U$, entonces:
\[ \id_A = U \varepsilon_1 \circ \eta_A = \id_A \circ \eta_A = \eta_A \]
Sea $f \colon B \to A$ morfismo cualquiera a $A$.
Por la naturalidad de $\eta$, se tiene que el siguiente diagrama conmuta:
\[\begin{tikzcd}
B \arrow[r,"\eta_B"] \arrow[d,"f" left] & A \arrow[d,"U(F f)"]\\
A \arrow[r,"\eta_A"] & A
\end{tikzcd}\]
Obsérvese que $U (F f) = U (\id_1) = \id_A$, luego tenemos que: $\eta_B = f$, lo que implica la unicidad de los morfismos $B \to A$ y que $A$ es objeto terminal.
\end{proof}

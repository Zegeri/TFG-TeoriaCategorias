\chapter*{Epílogo}

A estas alturas hemos visto las piezas básicas con las que se construye la teoría de categorías y cómo aparecen en programación funcional.
Sin embargo, la teoría de categoría es un campo mucho más grande que lo podemos abarcar en este trabajo.
Aquí damos un punto de comienzo para seguir el estudio de teoría de categorías.

En el capítulo 4, hemos identificado qué propiedades debe cumplir una categoría para que sea equivalente a la lógica proposicional intuicionista.
Añadiendo más propiedades a las categorías podemos obtener distintas lógicas.
Por ejemplo, podemos ``mejorar'' esta lógica a lógica intuicionista de orden superior.
Para ello, necesitaremos definir los \index{clasificador de subobjetos}\emph{clasificador de subobjetos}.
En pocas palabras, un subobjeto es cualquier objeto $S$ con un monomorfismo $S \to X$.
Un clasificador de subobjetos de $X$ es una colección de morfismos que caracterizan cada uno de sus subobjetos.
En pocas palabras, es una generalización de la idea de función característica de los conjuntos:
\begin{align*}
\chi_S & \colon X \to \{0,1\}\\
\chi_S (x) & = \begin{cases}0 & \text{si }x \notin S\\1 & \text{si }x \in S\end{cases}
\end{align*}
Una categoría cerrada cartesianamente con un clasificador de subobjetos se denomina \newterm{topos elemental}.
Resulta que un \newterm{topos elemental} es equivalente a la lógica intuicionista de orden superior.

En los capítulos 5, hemos introducido el concepto de monoide, pero también el concepto de categoría enriquecida y $2$-categoría.
Siguiendo esta idea llegamos a la teoría de categorías de orden superior. El libro de Lurie sirve de introducción a este tema.
Recordemos que en una $2$-categoría, tenemos $1$-morfismos entre $0$-morfismos (objetos) y $2$-morfismos entre $1$-morfismos.
Con esto en mente, definimos las $\infty$-categorías describiendo los $k$-morfismos entre $k-1$-morfismos.
Si además establecemos un particular concepto de equivalencia entre $k$-morfismos basado en nuestra descripción de equivalencias en la sección \ref{functores-adjuntos}, se introduce la idea de $(\infty,1)$-categorías, donde todos los $k$-morfismos son equivalentes para $k > 1$.
El estudio de estas $(\infty,1)$-categorías y su relación con la teoría de homotopía de tipos es un campo muy activo en los últimos años.
Véase el libro colaborativo 

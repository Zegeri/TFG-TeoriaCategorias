\chapter{Introducción}
La teoría de categorías nos ofrece al mismo tiempo un lenguaje común por el que se pueden entender matemáticos de distintas ramas a través de una poderosa abstracción a la composición de funciones.
Es más, la teoría de categorías ha encontrado un interés en el campo de la ciencia de la computación.
Para explorar la teoría de categorías haremos uso de la programación funcional.
Más concretamente, usaremos Haskell, que es uno de los lenguajes puramente funcionales más populares.

\section{Programación funcional}
La programación funcional es un estilo de programación que tiene sus orígenes en IPL (1956) y Lisp (1958).
Aunque durante mucho tiempo ha tenido un interés más académico que práctico, ha estado teniendo más atención en los últimos años.
Lenguajes de gran popularidad como C++ o Java han introducido características de la programación funcional en los últimos años.
\footnote{Since the introduction of the STL (Standard Template Library) -- about 1994 -- there has been a steady and cautious increase in the use of functional programming techniques in C++. -- Bjarne Stroustrup, creador de C++}
Algunos lenguajes, como Haskell o Standard ML, están diseñados con programación funcional en mente.
Encuentran su inspiración en las matemáticas y sus características principales son:

\begin{itemize}
  \item \emph{Funciones puras}. Las subrutinas no dejan efectos secundarios en memoria, por lo que sólo dependen en sus argumentos y no en el estado de la máquina.
  Por lo tanto, las funciones puras son más cercanas a las funciones con las que trabajan los matemáticos.
  \item \emph{Sistema de tipos fuertes}. El lenguaje categoriza los datos por tipos y restringe la aplicaciones de funciones a ciertos tipos, de manera que la expresión \code{"texto" + 1} da un error de compilación.
  A menudo hay un sistema de \emph{inferencia de tipo}, de manera que no es necesario tener que escribir los tipos constantemente.
  \item \emph{Funciones polimórficas}. En consecuencia del punto anterior y para permitir escribir funciones genéricas, se permite escribir funciones que actúen tipos arbitrarios. Por ejemplo, en Haskell, la declaración de tipo
  \begin{minted}{haskell}
  f :: forall a. a -> a
  \end{minted}
  describe \code{f} como una función que puede ser aplicado a todo tipo y devuelve el mismo tipo de salida.
  Una función equivalente en C++ sería de la forma:
  \begin{minted}{c++}
  template<class A>
  A f(A arg);
  \end{minted}
  Sin embargo, la programación funcional permite imponer restricciones a los tipos polimórficos que son validables en compilación a través de mecanismos como las \emph{clases de tipos} en Haskell.
  Son similares a las \emph{interfaces} de los lenguajes de programación orientada a objetos como Java.
  En algunos lenguajes funcionales, se permite además imponer restricciones de tipos al tipo polimórfico devuelto de una función, por ejemplo:
  \begin{minted}{haskell}
  read :: forall a. (Read a) => String -> a
  \end{minted}
  Esta función interpreta una cadena de texto como un tipo cualquiera \code{a} que sea de la clase de tipo \code{Read}.
  \item \emph{Tipos de datos algebraicos}. Pueden representar suma de tipos, además de producto de tipos.
  Esto permite la existencia de tipos definiciones recursivas como listas o árboles sin hacer uso de punteros.
  Además no son mutables (por el hecho de que las funciones no pueden tener efectos secundario), por lo que no hay una diferencia esencial entre los datos del constructor y los datos que almacena el tipo de datos.

  Por esta razón, la programación funcional suele resultar algo difícil para programadores acostumbrados a programar con efectos secundarios.
  Por otro lado, permite procesar los argumentos de una función por patrones.

  \item \emph{Evaluación perezosa}. Aunque no es característica de todos los lenguajes funcionales --por ejemplo, SML no lo incorpora--, muchos lenguajes funcionales como Haskell y Mirando lo adoptan.
  La evaluación perezosa permite que los argumentos de una función sean evaluados sólamente cuando sea necesario.
  Esto permite trabajar estructuras de datos infinitas, que en todo momento sólo están parcialmente definidas.
\end{itemize}

\section{Haskell}
Vamos a introducir lo mínimo de Haskell para poder entender el código que aquí se da.
No tendremos una imagen completa del lenguaje, pero sí lo suficiente para poder seguir el código.
Si ya está familiarizado con Haskell, puede saltarse esta sección.

Haskell nos permite \emph{declarar} una variable y su tipo:
\begin{minted}{haskell}
x :: Int
\end{minted}
y \emph{definir} el valor de dicha variable:
\begin{minted}{haskell}
x = 8
\end{minted}
Evidentemente, implementa aritmética básica:
\begin{minted}{haskell}
y = (x + 4) * 5
-- Esto es un comentario
-- y = 60
\end{minted}
Las funciones se definen de forma similar:
\begin{minted}{haskell}
duplica :: Int -> Int
duplica x = 2 * x
\end{minted}
Obsérvese que no se ponen paréntesis alrededor de los parámetros.
Tampoco se usan paréntesis cuando aplicamos la función, es decir, \code{duplica 3} es una expresión correcta.

Las funciones de varios parámetros se definen como funciones de orden superior:
\begin{minted}{haskell}
multiplica :: Int -> Int -> Int
multiplica x y = x * y
\end{minted}
Podemos entender que \code{multiplica} toma dos enteros y los multiplica, pero también se puede interpretar que \code{multiplica} se aplica a \code{x} y devuelve una función con tipo \code{Int -> Int}, que podemos a su vez aplicar al otro argumento.
Esta idea nos lleva al concepto de funciones parcialmente aplicadas:
\begin{minted}{haskell}
duplica' :: Int -> Int
duplica' x = multiplica 2 x
\end{minted}

Podemos definir funciones polimórficas, como explicamos en la sección anterior:
\begin{minted}{haskell}
max :: Ord a => a -> a -> a 
max x y =
  if x < y
  then x 
  else y
\end{minted}
Aquí vemos además la estructura condicional habitual \texttt{if-then-else}.

Trabajaremos a menudo a nivel de funciones, por lo que resultará útil echar un ojo al operador composición:
\begin{minted}{haskell}
(.) :: (b -> c) -> (a -> b) -> a -> c
f . g = \x -> f (g x)
\end{minted}
Toda función declarada entre paréntesis, como \code{(.)}, se corresponde con un operador infijo.
Por otro lado, la expresión $\backslash$\code{x -> f (g x)} representa una función de un sólo parámetro \code{x} y que devuelve \code{f (g x)}.
Más específicamente, representa la expresión lambda $\lambda x. f(g x)$.

Hablemos también un poco sobre los tipos de datos algebraicos.
Un tipo de dato se define con una expresión de la forma
\begin{minted}{haskell}
data Dato t1 ... tn = 
  Constructor1 a11 ... a1m | ... | ConstructorM aM1 ... aMq
\end{minted}
donde \code{t1 ... tn} son argumentos del tipo \code{Dato}, que tiene distintos constructores \code{Constructor1},\dots, \code{ConstructorM} cada uno con sus correspondientes parámetros.

Para entender mejor esto, veamos un ejemplo:
\begin{minted}{haskell}
data Maybe a = Nothing | Just a
\end{minted}
El tipo de dato \code{Maybe a} tiene dos constructores: \code{Nothing} (sin parámetro) y \code{Just a}, de parámetro \code{a}.
Recordemos que, en programación funcional, los constructores tienen toda la información sobre el tipo de dato, por lo que podemos identificar todo objeto del tipo \code{Maybe a} con uno de sus constructores.

Esto nos lleva a la interpretación que un objeto de tipo \code{Maybe a} es \code{Nothing} o bien \code{Just a}.
Pero cuidado, no confunda el constructor con el tipo.
Tanto \code{Nothing} como \code{Just 24} son del tipo \code{Maybe Int}.
Para definir una función que tome \code{Maybe a} por defecto, suele venir bien definirla por casos:
\begin{minted}{haskell}
maybe_a_numero :: Maybe Int -> Int
maybe_a_numero Nothing  = 0
maybe_a_numero (Just x) = x
\end{minted}

Otro tipo que nos resultará de utilidad es el tipo de listas \code{[a]}.
Esto tipo se salta algunas reglas de como está descrito para ser más sencillo de escribir.
Podemos definirlo en pseudo-haskell como:
\begin{minted}{haskell}
data [a] = a : [a] | []
\end{minted}
Es decir, una lista de elementos del tipo \code{a} es o bien una lista vacía \code{[]}, o bien un elemento de tipo \code{a} y otra lista del mismo tipo.

Podemos definir \emph{clases} sobre los tipos que implementen alguna función particular, por ejemplo:
\begin{minted}{haskell}
class Eq a where
  (==) :: a -> a -> Bool
\end{minted}
define la clase de los tipos sobre los que está definido el operador igualdad \code{(==)}.
Si queremos que un tipo esté en la clase \code{Eq}, tendremos que escribir su instancia.
Como ejemplo, supongamos que queremos escribir la instancia de \code{Maybe a} en la clase \code{Eq}.
Si los dos elementos del tipo \code{Maybe a} son de la forma \code{Nothing}, diremos que son iguales.
Si no, comparamos el argumento de \code{Just a}.
Pero para ello, el tipo \code{a} debe ser también de la clase \code{Eq}.
Por ello, una implementación sería:
\begin{minted}{haskell}
instance Eq a => Eq (Maybe a) where
  Nothing == Nothing = True
  Nothing == Just x  = False
  Just x  == Nothing = False
  Just x  == Just y  = x == y
\end{minted}

\section{Referencias}
\begin{itemize}
  \item Thompson, S. (1991). \emph{Type theory \& functional programming}, Capítulo 2.
  \item Pierce, B.C. (1991). \emph{A taste of category theory for computer scientist}, Capítulo 1.
  \item Lipovaca, M. (2011). \emph{Learn You a Haskell for Great Good!: A Beginner's Guide}.
\end{itemize}
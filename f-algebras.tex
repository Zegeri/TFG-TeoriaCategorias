\chapter{F-álgebras}
\section{Functores polinomiales}
Nuestro objetivo va a ser describir una cierta clase de endofunctores que pueden ser descritos por producto y coproductos.
Por esto, supongamos que estamos en una categoría $\cat$ con productos y coproductos finitos.
Recordemos que esto implica, en particular que existen objetos iniciales y finales.

\begin{definition}
Un \index{functor!polinomial}\emph{functor polinomial} es un miembro de la menor colección de functores que:
\begin{itemize}
  \item Contiene el functor identidad $\Id$.
  \item Contiene todo functor constante $\Delta_A$, que asocia todo elemento al objeto $A$.
  \item Es cerrada bajo composición, producto y coproducto, donde el producto de dos functores se corresponde con el functor:
  \[ (F \times G)(X) = F(X) \times G(X)\]
  y el coproducto se corresponde con:
  \[ (F + G)(X) = F(X) + G(X) \]
\end{itemize}
\end{definition}
Estos tipos de functores se expresan como un polinomio.
Por ejemplo, $F(X) = 1 + A \times X$ se debe entender como el functor que asocia a todo objeto $X$ el coproducto del producto de $A$ y $X$ con el objeto terminal $1$.

Consideremos el functor $F(X) = 1 + X$ y una $F$-álgebra $(A,f)$ donde $f \colon F(A) \to A$.
Por la definición de coproducto, este morfismo $f$ puede ser identificado con alguno de dos morfismos $f_1 \colon 1 \to U$ y $f_2 \colon A \to A$.
Viendo $F(X)$ como un tipo parametrizado por $X$, tiene sentido considerar $f_1$ y $f_2$ los \index{constructor}\emph{constructores} de $F(X)$.

Recordemos que en Haskell, el producto de dos tipos es su combinación en un par.
El objeto terminal se corresponde con \code{()} y el objeto inicial se define como
\begin{minted}{haskell}
data Empty
\end{minted}
Es decir, un tipo sin constructor.
En este contexto, podemos identificar $F(X)$ con el functor \code{Maybe a} y los morfismos $f_1$ y $f_2$ con \code{Nothing} y \code{Just}.

\section{F-álgebras}
\begin{definition}
Sea $\cat$ una categoría y $F \colon \cat \to \cat$ un endofunctor.
Una $F$-álgebra es un par $(A,\alpha)$, donde $A \in \cat$ y $\alpha \colon F a \to a$ es un morfismo de $\cat$.
\end{definition}

% A taste of category theory for computer scientist
En algunos libros, también se le llama \newterm{punto prefijo}.
Aunque no usaremos el nombre de punto prefijo, sí diremos que si $\alpha$ es isomorfismo, al par $(A,\alpha)$ par se le llama también \newterm{punto fijo}, señalando su similitud con la idea de punto fijo habitual de análisis.

Puede ser interesante darle una estructura de categoría a la colección de $F$-álgebras.
Para ello, necesitamos definir morfismos de un $(A,\alpha)$ a $(B,\beta)$.
Suponiendo que tenemos un morfismo $f \colon A \to B$, observamos que $F f \colon F A \to F B$, que puede ser compuesta con $\beta$:
\[ \beta \circ F f \colon F A \to B \]
Por otro lado:
\[ f \circ \alpha \colon F A \to B \]
La igualdad de estos dos morfismos nos dará la fundación de la categoría de $F$-álgebras.
Es más, como estos morfismos preservan bien la estructura de $F$-álgebras, les daremos el nombre (honorífico) de $F$-homomorfismo:

\begin{proposition}\label{prop:fmorfismo}
Sea $\cat$ una categoría y $F$ un endomorfismo en $\cat$.
Sea $Alg_F$ la colección de $F$-álgebras con homomorfismos $f \colon (A,\alpha) \to (B,\beta)$ tal que:
\begin{itemize}
\item $f \colon A \to B$ es un morfismo de $\cat$.
\item El siguiente diagrama conmuta
\[
\begin{tikzcd}
F A \arrow[r,"\alpha"] \arrow[d,"F f"] & A \arrow[d,"f"]\\
F B \arrow[r,"\beta"] & B
\end{tikzcd}
\]
\end{itemize}
Entonces $Alg_F$ forma una categoría.
\end{proposition}
\begin{proof}
Basta ver que la composición está bien definida, pues las demás condiciones se cumplen trivialmente.
Sean $f \colon (A,\alpha) \to (B,\beta)$ y $g \colon (B,\beta) \to (C,\gamma)$ homomorfismos de $F$-álgebras.
Entonces cada cuadrado del siguiente diagrama conmuta:
\[
\begin{tikzcd}
F A \arrow[r,"\alpha"] \arrow[d,"F f"] & A \arrow[d,"f"]\\
F B \arrow[r,"\beta"] \arrow[d,"F g"] & B \arrow[d,"g"]\\
F C \arrow[r,"\gamma"] & C
\end{tikzcd}
\]
Tenemos que:
\[ g \circ f \circ \alpha = g \circ \beta \circ F f = \gamma \circ F g \circ F f \]
Como $F g \circ F f = F (g \circ f)$, tenemos que el siguiente diagrama conmuta:
\[
\begin{tikzcd}
F A \arrow[r,"\alpha"] \arrow[d,"F (g \circ f)"] & A \arrow[d,"g \circ f"]\\
F C \arrow[r,"\gamma"] & C
\end{tikzcd}
\]
Luego $g \circ f$ es un homomorfismo de $F$-álgebras.
\end{proof}

Una vez hemos determinado una nueva categoría, podemos empezar a buscar qué aspecto tienen nuestra construcciones universales en esta nueva categoría.
Resultará de particular interés el objeto inicial de una $Alg_F$ por el siguiente lema debido a \hyperref[lambek-pf]{Lambek}:

\begin{lemma}\label{lambek-lemma}
Sea $F \colon \cat \to \cat$ un endofunctor.
Si $(A,\alpha)$ es una $F$-álgebra inicial de $Alg_F$, entonces $(A,\alpha)$ es un punto fijo de $F$.
\end{lemma}
\begin{proof}
Sea $(A,\alpha)$ un objeto inicial en $Alg_F$.
Para decir que $(A,\alpha)$ es un punto fijo, debemos ver que $\alpha$ es un isomorfismo.
Para ello consideramos el $F$-álgebra $(F A, F \alpha)$.
Como $(A,\alpha)$ es inicial, existe un único homomorfismo $u \colon (A, \alpha) \to (F A, F \alpha)$.

Por un lado, $\alpha \circ u \colon (A, \alpha) \to (A, \alpha)$, pero el único homomorfismo de un objeto inicial a sí mismo es el morfismo identidad, luego $\alpha \circ u = \id_{(A, \alpha)}$.
Por definición de homomorfismo, el siguiente diagram conmuta:
\[
\begin{tikzcd}
F A \arrow[r,"\alpha"] \arrow[d,"F u"] & A \arrow[d,"u"]\\
F (F A) \arrow[r,"F \alpha"] & F A
\end{tikzcd}
\]
Entonces:
\[ u \circ \alpha = F \alpha \circ F u = F (\alpha \circ u) = F (\id_{(A, \alpha)}) = \id_{(F A, F \alpha)} \]
Luego $\alpha$ tiene inversa y debe ser isomorfismo.
\end{proof}

Dada una $F$-álgebra inicial $(A,\alpha)$, para toda otra $F$-áĺgebra $(B,\beta)$, el único homomorfismo de $(A,\alpha)$ a $(B,\beta)$ recibirá el imponente nombre de \newterm{catamorfismo}.
Por ejemplo, el homomorfismo $u$ de la demostración era un catamorfismo a $(F A, F \alpha)$.

\section{Catamorfismos en Haskell}

Debemos empezar implementando álgebras en Haskell:
\begin{minted}{haskell}
type Algebra f a = f a -> a
\end{minted}
Hay que tener en cuenta que no estamos imponiendo que \code{f} sea un functor, ya que Haskell no permite imponer restricciones a constructores de datos sin recurrir a GADTs\footnote{\url{https://wiki.haskell.org/Data_declaration_with_constraint}}.

Lo siguiente que necesitamos es implementar el punto fijo de \code{f}:
\begin{minted}{haskell}
newtype Fix f = U (f (Fix f))
\end{minted}
Como \code{Fix} tiene un único constructor, hay un isomorfismo entre \code{Fix f} y \code{f (Fix f)}.
Este isomorfismo es \code{U}, y su inversa es:
\begin{minted}{haskell}
unFix :: Fix f -> f (Fix f)
unFix (U f) = f
\end{minted}

Ahora que sabemos que (\code{Fix f}, \code{U}) es el álgebra inicial, sabemos que existe un catamorfismo de este álgebra a todas las demás álgebras.
Consideramos cualquier otra álgebra (\code{b}, \code{g}), con \code{g :: Algebra f b}. Sea \code{m :: Fix f -> b} el correspondiente catamorfismo, entonces por \ref{prop:fmorfismo}:

\[
\begin{tikzcd}
{f (Fix f)} \arrow[r,"fmap\ m"] \arrow[d,"U"] & f b \arrow[d,"g"]\\
{Fix f} \arrow[r,"m"] & b
\end{tikzcd}
\]

Podemos invertir \code{U} con \code{unFix}, luego la conmutatividad del diagrama es equivalente a que:
\begin{minted}{haskell}
m = g . fmap m . unFix
\end{minted}
Usamos esto para definir recursivamente el catamorfismo:
\begin{minted}{haskell}
cata :: (Functor f) => (Algebra f b) -> (Fix f -> b)
cata g = g . fmap (cata g) . unFix
\end{minted}

Veamos un ejemplo con el functor \code{Maybe}
\begin{example}
Especialicemos el código anterior para \code{Maybe}:
\begin{minted}{haskell}
type AlgebraM a = Algebra Maybe a
type FixM = Fix Maybe
\end{minted}
Por ahora, no dice mucho. Pero al añadir las siguientes de líneas obtenemos una perspectiva interesante:
\begin{minted}{haskell}
type Nat = FixM

cero :: Nat
cero = U Nothing
suc :: Nat -> Nat
suc = U . Just
\end{minted}
Es decir, los naturales pueden verse como el punto fijo del functor \code{Maybe}.
\end{example}

\section{F-coálgebras en Haskell}
Una \index{F-coálgebra}\emph{$F$-coálgebra} es, naturalmente, el dual de una $F$-álgebra.
Más precisamente, es un par $(A,\alpha)$, donde $\alpha \colon A \to F A$.
En Haskell:
\begin{minted}{haskell}
type Coalgebra f a = a -> f a
\end{minted}

Al igual que con las álgebras, las $F$-coálgebras forman una categoría $Coalg_F$.
El lema de Lambek \eqref{lambek-lemma} en el contexto de las coálgebras dice así:
\begin{lemma}\ref{lambek-lemma}
Sea $F \colon \cat \to \cat$ un endofunctor.
Si $(A,\alpha)$ es una $F$-coálgebra terminal de $Coalg_F$, entonces $(A,\alpha)$ es un punto fijo de $F$.
\end{lemma}

El único homomorfismo de una $F$-coálgebra a la $F$-coálgebra terminal se denomina \newterm{anamorfismo}.
Como en el caso del catamorfismo, podemos definirlo recursivamente:
\begin{minted}{haskell}
ana :: (Functor f) => (Coalgebra f b) -> (b -> Fix f)
ana g = U . fmap (ana g) . g
\end{minted}

\begin{example}
Consideramos el functor $F(X)=1+A\times X$, que se puede ver como la composición de $G(X) = A \times X$ con $H(X) = 1+X$.
En términos de Haskell, la composición de \code{(a,)} con \code{Maybe}.
\begin{minted}{haskell}
data F a x = Maybe (a,x)
\end{minted}
Su punto fijo es un tipo de la forma $X = 1 + A \times X$, que se corresponde con el functor lista \code{[a]}.

Por lo tanto, en este contexto \code{ana} tiene la forma:
\begin{minted}{haskell}
ana :: (x -> Maybe (a,x)) -> (x -> [a])
\end{minted}
\code{ana} nos permite construir una lista (posiblemente infinita) a partir de una \textquote{semilla} \code{x} y una función que devuelva \code{a}, el valor generado, y \code{x}, una nueva semilla. 
Una implementación de este anamorfismo del paquete básico de Haskell es la función \code{unfoldr}.
\end{example}

\section{Referencias}
\begin{itemize}
  \item Pierce, B.C. (1991). \emph{A taste of category theory for computer scientist}, capítulo 3.4.
  \item Milewski, B. (2014). \emph{Category Theory for Programmers}, capítulo 24.
  \item Lambek, J. (1968). A fixed point theorem for complete categories. \emph{Mathematische Zeitschrift 103}, páginas 151--161. \label{lambek-pf}
\end{itemize}
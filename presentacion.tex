\documentclass{beamer}
\usepackage[utf8]{inputenc}
\usetheme{Berlin}
\usecolortheme{beaver}
\setbeamercolor{item projected}{bg=darkred}
\setbeamertemplate{enumerate items}[default]
\setbeamertemplate{navigation symbols}{}
\setbeamercovered{transparent}
\setbeamercolor{local structure}{fg=darkred}
\setbeamercolor{block title}{use=structure,fg=white,bg=red!75!black}
\setbeamercolor{block body}{use=structure,fg=black,bg=red!20!white}
\usepackage[spanish]{babel}
\usepackage{multirow}
\usepackage{bussproofs}
\usepackage{xcolor}
%\usepackage{estilo-apuntes}
\usepackage[]{graphicx}
\usepackage{tikz}
\usepackage{enumerate}
\usepackage{minted}
\usetikzlibrary{babel}
\usetikzlibrary{positioning, fit, shapes.geometric}
\usetikzlibrary{cd}
\theoremstyle{definition}

\newcommand{\cat}{{\mathcal{C}}}
\newcommand{\catD}{{\mathcal{D}}}
\newcommand{\Set}{{Set}}
\newcommand{\Grp}{{Grp}}
\newcommand{\Ab}{{Ab}}
\newcommand{\Vect}{{Vect}}
\newcommand{\Ring}{{Ring}}
\newcommand{\Cat}{{Cat}}
\newcommand{\Top}{{Top}}
\newcommand{\Hask}{{Hask}}

\newtheorem{teorema}{Teorema}
\newtheorem{defi}{Definición}
\newtheorem{prop}[teorema]{Proposición}
\newtheorem{idea}{Idea}
\newcommand{\Z}{\mathbb{Z}}
\newcommand{\C}{\mathbb{C}}
\newcommand{\D}{\mathbb{D}}
\providecommand{\gene}[1]{\langle{#1}\rangle}

\makeatletter
\newcommand*\bigcdot{\mathpalette\bigcdot@{.7}}
\newcommand*\bigcdot@[2]{\mathbin{\vcenter{\hbox{\scalebox{#2}{$\m@th#1\bullet$}}}}}
\makeatother

\DeclareMathOperator{\id}{id}

\addtobeamertemplate{navigation symbols}{}{%
    \usebeamerfont{footline}%
    \usebeamercolor[fg]{footline}%
    \hspace{1em}%
    \insertframenumber/\inserttotalframenumber
}
\setbeamercolor{footline}{fg=black}
\setbeamerfont{footline}{series=\bfseries}

%-----------------------------------------------------------

\title{Teoría de Categorías y Programación Funcional}
\author{Diego Pedraza López}
\institute{Universidad de Sevilla}
\date{}
 
\begin{document}
\frame{\titlepage}

%\begin{frame}
%\frametitle{Tabla de contenidos}
%\tableofcontents
%\end{frame}
%}

\section{Categorías}
\begin{frame}
\frametitle{¿Qué es una categoría?}
\begin{columns}
\begin{column}{0.5\textwidth}
  \begin{enumerate}
	\item<1-> Una colección de objetos\\$A, B, \dots$.
	\item<2-> Una colección de morfismos entre dichos objetos\\$f, g, \dots$.
	\item<3-> Una composición de morfismos\\$\circ$.
	\item<4-> Un morfismo $\id$ en cada objeto.
	\end{enumerate}
\end{column}
\begin{column}{0.5\textwidth}  %%<--- here
    \[
\begin{tikzcd}[row sep=3em,column sep=4.5em]
A
\onslide*<2,4>{\arrow{r}{}}
\onslide*<3>{\arrow[red]{r}{g}}
\onslide*<4->{\arrow[red,loop left]{}{\id_A}}
        \pgfmatrixnextcell 
B
\onslide*<4->{\arrow[red,loop right]{}{\id_B}}
\\
C
\onslide*<2->{\arrow{u}{}}
\onslide*<2->{\arrow{r}{}}
\onslide*<4->{\arrow[red,loop left]{}{\id_C}}
		\pgfmatrixnextcell
D
\onslide*<3>{\arrow[red,dashed]{u}[right]{g \circ f}}
\onslide*<4->{\arrow{u}{}}
\onslide*<2,4>{\arrow{ul}{}}
\onslide*<3>{\arrow[red]{ul}{f}}
\onslide*<4->{\arrow[red,loop right]{}{\id_D}}
\end{tikzcd}
\]
\end{column}
\end{columns}
\end{frame}

\begin{frame}
\frametitle{¿Qué es una categoría?}
\begin{enumerate}
\setcounter{enumi}{4}
\item<1->
\[ f \circ \id_A = f \]
\[ \id_B \circ f = f \]
\item<2->
\[ (f \circ g) \circ h = f \circ (g \circ h) \]
\end{enumerate}
\end{frame}

\begin{frame}
\frametitle{Un poco de notación}
\begin{columns}
\begin{column}{0.5\textwidth}
\[ \onslide*<1>{f \colon A \to B} 
\onslide*<2>{\mathcal{C}(A,B)} \]
\end{column}
\begin{column}{0.5\textwidth}  %%<--- here
\[
\onslide*<1>{\begin{tikzcd}[row sep=3em,column sep=4.5em]
A \arrow{r}{f} \pgfmatrixnextcell  B
\end{tikzcd}}
\onslide*<2>{\begin{tikzcd}[row sep=3em,column sep=4.5em]
A \arrow{d}{}\arrow[bend right=20]{d}{} \arrow[bend right=40]{d}{} \arrow[bend left=20]{d}{\dots}\\  B
\end{tikzcd}}
\]
\end{column}
\end{columns}
\end{frame}

\begin{frame}
\frametitle{Algunos ejemplos}
\begin{columns}
\begin{column}{0.5\textwidth}
  \begin{enumerate}
	\item<1-> $\mathbf{\Set}$.
	\item<2-> $\mathbf{\Grp}$.
	\item<3-> $\mathbf{\Top}$.
	\item<4-> $\mathbf{1}$.
	\item<5-> $\mathbf{\Cat}$.
	\end{enumerate}
\end{column}
\begin{column}{0.5\textwidth}  %%<--- here
    \[
\onslide*<1,2,3,5>{
\begin{tikzcd}[row sep=3em,column sep=4.5em]
{\onslide*<1>{\{1,2\}}
\onslide*<2>{\Z_2 \times \Z_2}
\onslide*<3>{\mathbb{R}^4}
\onslide*<5>{\Grp}}
\arrow{r}{}
        \pgfmatrixnextcell 
{\onslide*<1>{\mathbb{R}}
\onslide*<2>{C_4}
\onslide*<3>{\mathbb{C}}
\onslide*<5>{\mathbf{1}}
}
\\
{\onslide*<1>{\emptyset}
\onslide*<2>{\{1\}}
\onslide*<3>{[0,1)}
\onslide*<5>{\Set}}
\arrow{u}{}
		\pgfmatrixnextcell
{\onslide*<1>{[0,1]}
\onslide*<2>{S_8}
\onslide*<3>{\mathbb{R}^2}
\onslide*<5>{\Top}}
\arrow{u}{}
\arrow{ul}{}
\end{tikzcd}}
\onslide*<4>{
\begin{tikzcd}[row sep=3em,column sep=4.5em]
{*}
\arrow[loop above]{}{\id_*}
\end{tikzcd}
}
\]
\end{column}
\end{columns}
\end{frame}

\begin{frame}
\frametitle{Functores: Aplicaciones entre categorías}

\[
\begin{tikzcd}[row sep=3em,column sep=4.5em]
\cat \arrow{r}{F} \pgfmatrixnextcell \catD
\end{tikzcd}
\]
\[
\begin{tikzcd}[row sep=3em,column sep=4.5em]
A \arrow[mapsto]{r}{}\onslide*<2>{\arrow{d}{f}}\pgfmatrixnextcell {F A} \onslide*<2>{\arrow{d}{F f}}\\
\onslide*<1>{\ }\onslide*<2>{B \arrow[mapsto]{r}{}} \pgfmatrixnextcell 
\onslide*<1>{\ \ }\onslide*<2>{F B}
\end{tikzcd}
\]
\end{frame}

\section{Programación Funcional}
\begin{frame}
\frametitle{¿Qué tienes que ver esto con la programación?}

\begin{center}
Teoría de tipos de la programación funcional
\end{center}
\end{frame}

\begin{frame}
\frametitle{¿Programación funcional?}
\begin{itemize}
\item<2-> Funciones puras.
\item<3-> Sistema de \textbf{tipos}. \color{gray}{Cálculo lambda con tipos simples}
\item<4-> Funciones polimórficas. \color{gray}{System F}
\item<5-> Y otras cosas.
\end{itemize}
\end{frame}

\begin{frame}[fragile]
\frametitle{Haskell}
\begin{minted}{haskell}
--         Tipo     Tipo
repite_a :: Int -> String
repite_a x = repeat x 'a'
\end{minted}
\end{frame}

\begin{frame}[fragile]
\frametitle{La categoría Hask}
\begin{center}
\begin{tikzcd}[row sep=3em,column sep=6em]
\color{red}{\text{Int}}
\arrow[bend left]{r}{\color{blue}{\text{repite\_a}}}
\onslide*<2>{\arrow{r}{\color{blue}{\text{repite\_b}}}
\arrow[bend right]{r}{\color{blue}{\text{repite\_c}}}
\arrow[bend right=80]{r}{\vdots}}
\pgfmatrixnextcell \color{red}{\text{String}}
\end{tikzcd}
\end{center}
\end{frame}

\begin{frame}[fragile]
\frametitle{Composición en Hask}
\begin{minted}{haskell}
repite_a :: Int -> String

longitud :: String -> Int

composicion :: Int -> Int
composicion = longitud . repite_a
\end{minted}
\end{frame}

\section{Construcciones sobre categorías}
\begin{frame}
\begin{center}Volvamos a las categorías.\end{center}
\end{frame}

\begin{frame}
\frametitle{Isomorfismos}
\[ 
\begin{tikzcd}[row sep=3em, column sep=4.5em]
A \arrow[bend left]{r}{f}\pgfmatrixnextcell B \arrow[bend left]{l}{g}
\end{tikzcd}
\]

\[ \onslide*<2->{f \circ g = \id_B \quad g \circ f = \id_A} \]

\[ \onslide*<3->{A \cong B}\]
\end{frame}

\begin{frame}
\frametitle{Objeto terminal}
\[
\begin{tikzcd}[row sep=3em,column sep=2em]
\pgfmatrixnextcell \pgfmatrixnextcell 1\\
\bigcdot \arrow{urr}{} \pgfmatrixnextcell \bigcdot \arrow{ur}{} \pgfmatrixnextcell \bigcdot \arrow{u}{} \pgfmatrixnextcell \bigcdot \arrow{ul}{} \pgfmatrixnextcell  \dots
\end{tikzcd}
\]
\end{frame}


\begin{frame}
\begin{idea}[Dualizar]
Cambiar de dirección los morfismos.
\end{idea}
\end{frame}

\begin{frame}
\frametitle{Objeto inicial}
\[
\begin{tikzcd}[row sep=3em,column sep=2em]
\pgfmatrixnextcell \pgfmatrixnextcell 0 \arrow{dll}{} \arrow{dl}{} \arrow{d}{} \arrow{dr}{}\\
\bigcdot  \pgfmatrixnextcell \bigcdot  \pgfmatrixnextcell \bigcdot  \pgfmatrixnextcell \bigcdot  \pgfmatrixnextcell  \dots
\end{tikzcd}
\]
\end{frame}

\begin{frame}
\frametitle{Objeto producto}
\[\begin{tikzcd}[row sep=3em,column sep=4.5em]
 \pgfmatrixnextcell A\\
\onslide*<2->{V \arrow{ur}{f_A} \arrow{dr}[below]{f_B}} \onslide*<3>{\arrow[dashed]{r}{h}} \pgfmatrixnextcell {A \times B} \arrow{u}[right]{p_A} \arrow{d}{p_B}\\
 \pgfmatrixnextcell B
\end{tikzcd}\]
\end{frame}

\begin{frame}
\frametitle{Objeto coproducto}
\[\begin{tikzcd}[row sep=3em,column sep=4.5em]
 \pgfmatrixnextcell A \arrow{d}{i_A}\onslide*<2->{\arrow{dl}[above]{f_A}}\\
\onslide*<2->{V} \pgfmatrixnextcell {A + B} \onslide*<2->{\arrow[dashed]{l}{h}}\\
 \pgfmatrixnextcell B \arrow{u}[right]{i_B}\onslide*<2->{\arrow{ul}{f_B}}
\end{tikzcd}\]
\end{frame}

\begin{frame}
\frametitle{Objeto exponencial}
\[
\begin{tikzcd}[row sep=3em,column sep=4.5em]
C^B \times B \arrow{r}{\varepsilon} \pgfmatrixnextcell C \\
\onslide*<2->{A \times B  \arrow{ur}[below]{f}} \onslide*<3>{\arrow[dashed]{u}{\lambda_f\times\id_B}}
\end{tikzcd}
\]
\end{frame}

\section{Categorías Cartesianas Cerradas}
\begin{frame}
\frametitle{Categorías Bicartesianas Cerradas}
Una \textbf{categoría bicartesiana cerrada} (CBC) tiene:
\begin{itemize}
\item<2-> Objeto inicial y terminal.
\item<3-> Productos y coproductos.
\item<4-> Exponenciales.
\end{itemize}
\end{frame}

\begin{frame}[fragile]
\frametitle{Hask es CBC}
\begin{itemize}
\item<1-> Objeto inicial
\begin{minted}{haskell}
data Empty
\end{minted}
\item<2-> Objeto terminal
\begin{minted}{haskell}
() :: ()
\end{minted}
\item<3-> Producto
\begin{minted}{haskell}
(a, b)
\end{minted}
\item<4-> Coproducto
\begin{minted}{haskell}
data Either a b = Left a | Right b
\end{minted}
\item<5-> Exponencial
\begin{minted}{haskell}
curry :: ((a, b) -> c) -> (a -> b -> c)
\end{minted}
\end{itemize}
\end{frame}

\begin{frame}
\begin{idea}[Lambek]
CBC es lógica proposicional intuicionista  (LPI).
\end{idea}
\end{frame}

\begin{frame}
\frametitle{Lógica Proposicional Intuicionista}
La LPI es un grafo de \textbf{fórmulas} con \textbf{pruebas} como flechas.
\begin{enumerate}
\itemsep1em 
\item<1-> $A \to A$.
\onslide*<2->{$\cong$ Morfismo identidad}
\item<3-> 
\AxiomC{$A \to B$}
\AxiomC{$B \to C$}
\BinaryInfC{$A \to C$}
\DisplayProof
\onslide*<4->{$\cong$ Composición de morfismos}

\item<5-> $A \to \top$. 
\onslide*<6->{$\cong$ Objeto terminal}
\end{enumerate}
\end{frame}

\begin{frame}
\frametitle{Lógica Proposicional Intuicionista}
\begin{enumerate}
\itemsep1em
\setcounter{enumi}{3}
\item $A \land B \to A$.
\item $A \land B \to B$.
\item
  
\AxiomC{$C \to A$}
\AxiomC{$C \to B$}
\BinaryInfC{$C \to A \land B$}
\DisplayProof

\onslide*<2->{$\cong$ Objeto producto}
\end{enumerate}
\end{frame}

\begin{frame}
\frametitle{Lógica Proposicional Intuicionista}
\begin{enumerate}
\itemsep1em
\setcounter{enumi}{6}
\item $(A \Rightarrow B) \land A \to B$.
\item 

  \AxiomC{$C \land A \to B$}
  \UnaryInfC{$C \to A \Rightarrow B$}
  \DisplayProof

\onslide<2>{$\cong$ Exponencial}
\end{enumerate} 
\end{frame}

\begin{frame}
\frametitle{Lógica Proposicional Intuicionista}
\begin{enumerate}
\itemsep1em
\setcounter{enumi}{8}
\item<1-> $\bot \to A$. \onslide*<2->{$\cong$ Objeto inicial}
\item<3-> $A \to A \lor B$
\item<3-> $B \to A \lor B$
\item<3->
\AxiomC{$A \to C$}
\AxiomC{$B \to C$}
\BinaryInfC{$A \lor B \to C$}
\DisplayProof

\onslide*<4->{$\cong$ Coproducto}
\end{enumerate} 
\end{frame}

\begin{frame}[fragile]
\frametitle{Negación}
\[ \neg p \equiv p \Rightarrow \bot \]

\[ \neg A \equiv 0^A \]

\begin{minted}{haskell}
              type Not a = a -> Empty
\end{minted}

\end{frame}
\begin{frame}[fragile]
\frametitle{Demostración "por escritura"}
\[ \neg p \lor \neg q \to \neg (p \land q) \]

\begin{minted}{haskell}
de_morgan :: Either (Not a) (Not b) -> Not (a, b)
de_morgan (Left f) = \(x, y) -> f x
de_morgan (Right f) = \(x, y) -> f y
\end{minted}
\end{frame}
\setbeamertemplate{headline}{}
\section*{}
\begin{frame}
\begin{figure}[h!]
Muchas gracias
\end{figure}
\end{frame}
\end{document}
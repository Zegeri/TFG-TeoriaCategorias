\chapter{Mónadas}
\section{Mónadas de una categoría}
Dada una categoría $\cat$, consideramos la categoría $\cat^\cat$, donde los objetos son endofunctores y los morfismos vienen dados por la composición entre functores.

\begin{proposition}
$\cat^\cat$ forma una categoría monoidal con la composición como producto tensorial y functor unidad $\Id$ como objeto unidad.
\end{proposition}
\begin{proof}
Esto es consecuencia de que $\Cat$ sea una $2$-categoría, como vimos en \ref{cat-2-categoria}.
\end{proof}

Por lo tanto, podemos definir monoides sobre $\cat^\cat$.
Dichos monoides se denominan \newterm{mónadas}.
Tratemos de dar una definición más explícita, especializando la definición de monoide teniendo en cuenta que:
\begin{itemize}
  \item $\circ$ es asociativa.
  \item El functor identidad $\Id \colon \cat \to \cat$ cumple que:
\[ \Id \circ F = F \]
\[ F \circ \Id = F\]
  \item El bifunctor $\circ$ actuando sobre transformaciones naturales es la composición horizontal $*$.
  \item Para la transformación natural identidad $\iota \colon F \Rightarrow F$:
\[ \mu * \iota = \mu F \]
\[ \iota * \mu = F \mu \]
  \item Escribiremos $F^n$ para referirnos a $\underbrace{F \circ \dots \circ F}_{n \text{ veces}}$.
\end{itemize}

\begin{definition}
Una \emph{mónada} es un endofunctor $F$ de una categoría $\cat$ con dos transformaciones naturales:
\begin{tikzcd}
F^2 \arrow[r,"\mu"] & F & \Id \arrow[l,"\eta" above]
\end{tikzcd}
que cumplan:
\begin{itemize}
  \item La propiedad asociativa:
  \[ \begin{tikzcd}
  &  {F^3} \arrow[dl,"\mu F" above,Rightarrow] \arrow[dr,"F\mu",Rightarrow]\\
  {F^2} \arrow[dr,"\mu",Rightarrow] & & {F^2} \arrow[dl,"\mu" above,Rightarrow]\\
  & F
  \end{tikzcd}\]
  \item La propiedad de la unidad:
  \[ \begin{tikzcd}
  F \arrow[r,"\eta F",Rightarrow] \arrow[dr,"\iota" below,Rightarrow] & {F^2} \arrow[d,"\mu",Rightarrow] & {F} \arrow[l,"F\eta" above,Rightarrow] \arrow[dl,"\iota",Rightarrow]\\
  & F
  \end{tikzcd}\]
\end{itemize}
\end{definition}


%Categories and Haskell 7
\begin{example}
Sea $\mathcal{P} \colon \Set \to \Set$ el endofunctor potencia y consideramos las transformaciones naturales.
\begin{align*} \mu_A \colon \mathcal{P}^2(A) & \to \mathcal{P}(A)\\
\{S_i\}_{i \in I} & \mapsto \bigcup_{i \in I} B_i
\end{align*}
\begin{align*} \eta_A \colon \Id A = A & \to \mathcal{P}(A)\\
a & \mapsto \{a\}
\end{align*}

Veamos que cumple las condiciones de mónada. 
Sea $A$ un conjunto cualquiera.
Obsérvese que $\mu_{\mathcal{P}(A)}(\mathcal{P}(A)) = A$, pues $A \in \mathcal{P}(A)$ y todo elemento de $\mathcal{P}(A)$ está contenido en $A$.
\begin{align*}
(\mu \circ \mu \mathcal{P})(\mathcal{P}^3(A)) & = \mu_{\mathcal{P}^2(A)}\left(\mu_{\mathcal{P}^3(A)}\mathcal{P}^3(A)\right)\\
& = \mu_{\mathcal{P}^2(A)}\mathcal{P}^2(A)\\
& = \mathcal{P}(A)
\end{align*}
\begin{align*}
(\mu \circ \mathcal{P}\mu)(\mathcal{P}^3(A)) & = \mu_{\mathcal{P}^2(A)}\left((\mathcal{P}\mu_{\mathcal{P}^2(A)})\mathcal{P}^3(A)\right)\\
& = \mu_{\mathcal{P}^2(A)}\mathcal{P}^2(A)\\
& = \mathcal{P}(A)
\end{align*}
\begin{align*}
(\mu \circ \eta\mathcal{P})(\mathcal{P}(A)) & = \mu_{\mathcal{P}^2(A)}\eta_{\mathcal{P}(A)}\mathcal{P}(A)\\
& = \mu_{\mathcal{P}^2(A)}(\{\mathcal{P}(A)\})\\
& = \mathcal{P}(A)
\end{align*}
\begin{align*}
(\mu \circ \mathcal{P}\eta)(\mathcal{P}(A)) & = \mu_{\mathcal{P}^2(A)} \left(\mathcal{P}\eta_{\mathcal{P}(A)}(\mathcal{P}(A))\right)\\
& = \mu_{\mathcal{P}^2(A)} \left(\{\mathcal{P}(A)\}\right)\\
& = \mathcal{P}(A)
\end{align*}
\end{example}

%Categories and Haskell + Awodey 10
\section{Categoría de Kleisli}
La siguiente construcción nos llevará de vuelta al dominio de Haskell.
\begin{definition}
Sea $(F,\eta,\mu)$ una mónada sobre una categoría $\cat$.
La \index{categoría!de Kleisli}\emph{categoría de Kleisli} $\cat_F$ es la categoría donde:
\begin{itemize}
  \item Sus objetos son los objetos de $\cat$.
  \item Un morfismo $A \to_F B$ en $\cat_F$ es un morfismo de la forma $A \to F B$ en $\cat$.
  \item La composición entre dos morfismos $f \colon A \to_F B$ y $g \colon B \to_F C$ viene dado por la conmutatividad del siguiente diagrama:
  \[ \begin{tikzcd}
  A \arrow[r,"g \circ_F f"] \arrow[d,"f" left] & {F C}\\
  {F B} \arrow[r,"F g" below] & F^2 C \arrow[u,"\mu_C" right]
  \end{tikzcd}\]
  \item El morfismo identidad $\id_A$ en $\cat_F$ se correspoonde con $\eta_A$ en $\cat$.
\end{itemize}
\end{definition}
Para demostrar que la categoría de Kleisli es realmente un categoría, nos limitamos a comprobar la asociatividad:

Para $h \colon C \to F D$, $g \colon B \to F C$ y $f \colon A \to F B$, se tiene:
\begin{align*}
(h \circ_F g) \circ f = (\mu_D \circ F h \circ g) \circ_F f\\
& = \mu_D \circ F(\mu_D \circ F h \circ g)\circ g\\
& = \mu_D \circ F\mu_D \circ F^2 h \circ F g \circ g
\end{align*}
Por la naturalidad de $\mu$.

\section{Mónadas en Haskell}
¿Cómo vuelve todo esto a la programación funcional?
Consideremos el siguiente ejemplo:
\begin{minted}{haskell}
convertir :: String -> Maybe Float
inversa :: Float -> Maybe Float
\end{minted}
donde \code{convertir} intenta convertir una cadena de texto a un número e inversa intenta dar la inversa de un número.
En estos dos ejemplos, hay casos donde la función fallaría.
Cuando \code{String} no se puede convertir a un número, o cuando \code{inversa} intenta invertir el número \code{0}.
Imaginemos que quisieramos combinar estas dos funciones en una nueva función:
\begin{minted}{haskell}
convertir_inversa :: String -> Maybe Float
\end{minted}
Donde \code{convertir_inversa} primero usa \code{convertir} y luego usa \code{inversa} sobre el posible resultado.
¿Cómo podríamos implementar esta función?
Un camino sería el siguiente:
\begin{minted}{haskell}
convertir_inversa str =
  let conv = convertir str in
  if isNothing conv
  then Nothing
  else inversa (fromJust conv)
\end{minted}

Recuérdese que \code{Maybe} es un functor, luego podemos aplicar \code{Maybe} a la aplicación inversa para obtener:
\begin{minted}{haskell}
fmap inversa :: Maybe Float -> Maybe (Maybe Float)
\end{minted}
ahora que el dominio de \code{fmap inversa} coincide con el codominio de \code{convertir}, podemos hacer su composición:
\begin{minted}{haskell}
fmap inversa . convertir :: String -> Maybe (Maybe Float) 
\end{minted}
Aquí entra en juego la transformación natural $\mu \colon F^2 \Rightarrow F$, que en Haskell recibe el nombre de \code{join}, de tipo 
\begin{minted}{haskell}
join :: Monad m => m (m a) -> m a
\end{minted}
\code{join} nos permite transformar \code{Maybe (Maybe Float)} en \code{Maybe Float}.
Luego podríamos implementar \code{convertir_inversa} como la composición de estas tres funciones:
\begin{minted}{haskell}
convertir_inversa = join . fmap inversa . convertir
\end{minted}
Estamos definiendo \code{convertir_inversa} de manera que el siguiente diagram conmute:
\begin{displaymath}\begin{tikzcd}[column sep=6em,row sep=4em]
{String} \arrow[r,"convertir\_inversa",dashed] \arrow[d,"inversa"]& {Maybe\ Float}\\
{Maybe\ String} \arrow[r,"{fmap\ inversa}"] & {Maybe\ (Maybe\ Float)} \arrow[u,"join"]
\end{tikzcd}\end{displaymath}
Se puede reconocer aquí el diagrama que define la composición de Kleisli $\circ_F$.
En Haskell, dicha composición se denota a menudo como \code{>=>}, llamado conmúnmente \newterm{operador pez}, con tipo:
\begin{minted}{haskell}
(>=>) :: Monad m => (a -> m b) -> (b -> m c) -> a -> m c
\end{minted}
y nos permite dar la siguiente definición sucinta de \code{convertir_inversa}
\begin{minted}{haskell}
convertir_inversa = convertir >=> inversa
\end{minted}

Las mónadas en Haskell están definidas como una subclase de \code{Applicative}, que a su vez es una subclase de \code{Functor}.
Aunque no entraremos mucho en ello, \code{Applicative} representa los functores que mantienen la estructura monoidal de $\Hask$, y están definidos como:
\begin{minted}{haskell}
class Functor f => Applicative f where
    pure :: a -> f a
    (<*>) :: f (a -> b) -> f a -> f b
\end{minted}

Las mónadas se definen por la clase \code{Monad}:
\begin{minted}{haskell}
class Applicative m => Monad m where
  (>>=)       :: m a -> (a -> m b) -> m b
  return      :: a -> m a
\end{minted}
Además, se deben cumplir las llamadas leyes de mónadas:
\begin{minted}{haskell}
return v >>= k = k v
x >>= return = x
m >>= (\y -> k y >>= h) = (m >>= k) >>= h
\end{minted}

El operador \code{(>>=)} se puede definir en terminos de \code{join :: m (m a) -> m a}, que era la implementación de la transformación natural $\mu$, y viceversa.
\begin{minted}{haskell}
x >>= f = join (fmap f x)
join x = x >>= id
\end{minted}
También se puede implementar el operador pez de la categoría de Kleisli \code{(>=>)} con \code{(>>=)}.
\begin{minted}{haskell}
f >=> g = \x -> f x >>= g
\end{minted}

Haskell añade una forma de hacer operaciones con mónadas de forma aparentemente imperativa:
\begin{minted}{haskell}
convertir_inversa x = do
  y <- convierte x
  inversa y
\end{minted}
Esta expresión es equivalente a:
\begin{minted}{haskell}
convertir_inversa x =
  convierte x >>= \y ->
    inversa y
\end{minted}

Hemos usado implícitamente que \code{Maybe} forma una mónada, comprobémoslo:
\begin{example}
El functor \code{Maybe} forma una mónada:
\begin{minted}{haskell}
instance Monad Maybe where
  (Just x) >>= f = f x
  Nothing >>= _ = Nothing

  return x = Just x
\end{minted}
Veamos que cumple las leyes de las mónadas:
\begin{minted}{haskell}
  return v >>= k 
= { definición de return }
  Just x >>= k
= { definición de (>>=) }
  k x
\end{minted}
Esto demuestra la primera ley.
Demostraremos las otras dos leyes por casos:
\begin{minted}{haskell}
  Nothing >>= return
= { definición de (>>=) }
  Nothing
\end{minted}
\begin{minted}{haskell}
  (Just x) >>= return
= { definición de (>>=) }
  return x
= { definición de return }
  Just x
\end{minted}
Esto demuestra la segunda ley.
Finalmente:
\begin{minted}{haskell}
  Nothing >>= (\y -> k y >>= h)
= { definición de (>>=) }
  Nothing
= { definición de (>>=) }
  Nothing >>= h
= { definción de (>>=) }
  (Nothing >>= k) >>= h
\end{minted}
\begin{minted}{haskell}
  (Just x) >>= (\y -> k y >>= h)
= { definición de (>>=) }
  (\y -> k y >>= h) x
= { reducción beta }
  k x >>= h
= { definición de (>>=) }
  (Just x >>= k) >>= h
\end{minted}
Por lo que \code{Maybe} forma una mónada.
\end{example}

\section{Referencias}
\begin{itemize}
  \item Awodey, S. (2006). \emph{Category Theory}, Capítulos 7 y 10.
  \item Buurlage, J. (2017). \emph{Categories and Haskell}, Capítulo 7.
  \item Milewski, B. (2014). \emph{Category Theory for Programmers}, Capítulo 20.
\end{itemize}
